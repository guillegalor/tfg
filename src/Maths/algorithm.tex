\chapter{Algoritmo para códigos Reed-Solomon sesgados}%
\label{chap:algoritmo}

A lo largo de este capítulo mostraremos los resultados inmediatamente necesarios para construir los algoritmos principales de este trabajo. El desarrollo que hemos seguido se encuentra basado en~\cite{sugi}.

\section{Introducción}

En este capítulo seguiremos la notación introducida en el capítulo anterior, por tanto \(\F\) será un cuerpo cualquiera,  \(\sigma\) un autormorfismo de  \(\F\) de orden finito  \(n\), y \(R = \F[x;\sigma]\) el anillo de polinomio sesgados correspondiente.

En primer lugar, utilizaremos la identificación análoga a la descrita en~\ref{sec:códigos_cíclicos} entre vectores \(c = c_0 c_1\cdots c_{n-1}\) de \(\F^n\) y polinomios \(c(x) = c_0 + c_1x + \cdots + c_{n-1}x^{n-1}\) de \(\F[x;\sigma]\). Ahora, si queremos seguir los pasos que utilizamos en dicha sección, queremos construir el anillo cociente \(F[x;\sigma]/\langle x^{n}-1\rangle\), y para ello necesitamos que el polinomio \(x^{n} - 1\) sea central en \(R\). Efectivamente, dado \(f = f_m x^m + \ldots + f_1x + f_0 \in \F\), tenemos que

\[
\begin{aligned}
(x^n-1)f &= x^n f_m x^m - f_m x^m + \dots + x^n f_1 x - f_1 x + x^n f_0 - f_0 \\
&= \sigma^n(f_m)x^{n+m} - f_m x^m + \dots + \sigma^n(f_1)x^{n+1} - f_1 + \sigma^n(f_0)x^n - f_0\\
&= f_mx^{n+m} - f_m x^m + \dots + f_1x^{n+1} - f_1 + f_0x^n - f_0\\
&= f(x^n -1).
\end{aligned}
\]

Por tanto el ideal por la izquierda que genera es también un ideal por la derecha, así que podemos considerar el anillo cociente \(\mathcal{R} = \F[x;\sigma] / \langle x^n -1 \rangle\). Entonces, cada ideal a la izquierda \(I \leq \mathcal{R}\) designa un código \(\C =\mathfrak{v}(I)\) de longitud  \(n\), donde \(\mathfrak{v}:\mathcal{R} \rightarrow \F^n\) es el mapa de coordenadas asociado a la base \(\mathcal{B} = \{1, x, \dots, x^n\}\). Así, la longitud del código coincide con el orden de \(\sigma\). De aquí en adelante suponemos establecidas estas condiciones. Llamaremos a cualquier código de este tipo \textit{código cíclico sesgado}, o CCS para abreviar.

Recordemos la proposición TODO que nos dice que el centro de \(\F[x;\sigma]\) es el anillo de polinomios conmutativo  \(\F^\sigma[x^n]\), donde  \(\F^\sigma\) denota el subcuerpo invariante por  \(\sigma\), es decir, los elementos  \(a \in \F\) tales que  \(\sigma(a) = a\). Para continuar necesitamos conocer mejor la estructura de nuestro anillo \(\mathcal{R}\), y para demostraremos el siguiente teorema.

% Notas Isomorfía con Anillo de Matrices
% R(x-1) \in R
% R/R(x-1) ann as R-module = <x^n -1> // Haces cociente sobre el anulador y sigue siendo simple
% R/Rx^n-1
% El anillo cociente por el anulador es un algebra sobre los invariantes.

% ---

Veamos a continuación un método para construir CCSs. Sabemos que todo ideal a la izquierda de \(\mathcal{R}\) es principal, pues ya dijimos que los anillos de polinomios sesgados son dominios de ideales principales a la izquierda, así que todo CCS está generado por un divisor a la derecha de \(x^n - 1\). Por tanto, de forma análoga a como se hace para polinomios cíclicos, necesitaremos encontrar factores a la derecha de este. El problema que nos encontramos es que, hasta donde sabemos, no existe un algoritmo de factorización completa para polinomios de Ore sobre un cuerpo cualquier. Por tanto construiremos un procedimiento específico para \(x^n - 1\). Veamos primero un método para encontrar divisores lineales a la derecha.

\begin{proposition}
\label{prop:x_beta}
    Sea \(\beta \in \F\), entonces  \(x - \beta\) divide por la derecha a  \(x^n -1\) si y solo si  \(\beta = \sigma(c)c^{-1}\) para algún  \(c \in \F\) distinto de cero.
\end{proposition}

\begin{proof}
    Sea \(R = \F[x; \sigma]\). Si \(x - \beta\) divide por la derecha a \(x^n - 1\), entonces \(R/R(x - \beta) \cong R/R(x-1)\) como \(\mathcal{R}\)-módulos. Por la (TODO referencia), tenemos \(\beta = \sigma(c)c^{-1}\) para algún \(c \in \F\) distinto de 0. Para la ver la implicación contraria, sea  \(c \in \F\), y  \(\beta = \sigma(c)c^{-1}\) \dots (TODO).
\end{proof}

Por el teorema \ref{th:matrix_iso}, sabemos que \(x^n - 1\) se puede descomponer como el mínimo común múltiplo por la izquierda de polinomios lineales (correspondiendo al ideal cero de \(\mathcal{R}\) visto como intersección de \(n\) submódulos a la izquierda maximales (TODO)). Con el objetivo de encontrar una descomposición de este tipo de \(x^n - 1\) (y por tanto divisores a la derecha no lineales (TODO)), nuestra estrategia es construir \(\beta \in \F\) tal que
\begin{equation}
\label{eq:lclm}
[x - \beta, x - \sigma(\beta), \dots, x - \sigma^{n-1}(\beta)]_l = x^n -1
\end{equation}

Por analogía con~\cite[Proposición 1]{SCC_boucher} es claro que
\ref{eq:lclm} se cumple si y solo el determinante de la matriz

\[
\begin{pmatrix}
    1 & \beta & \beta\sigma(\beta) & \dots & \beta\sigma(\beta)\dots\sigma^{n-2}(\beta) \\
    1 & \sigma(\beta) & \sigma(\beta)\sigma^2(\beta) & \dots & \sigma(\beta)\sigma^2(\beta)\dots\sigma^{n-1}(\beta) \\
    \vdots & \vdots & \vdots & \ddots & \vdots \\
    1 & \sigma^{n-1}(\beta) & \sigma^{n-1}(\beta)\beta & \dots & \sigma^{n-1}(\beta)\beta\dots\sigma^{n-3}(\beta)

\end{pmatrix}
\]

es distinto de cero. Utilizando que \(\beta = \sigma(c)c^{-1}\) por la proposición \ref{prop:x_beta}, esto es equivalente a que el determinante de la matriz

\[
\begin{pmatrix}
    c & \sigma(c) & \sigma^2(c) & \dots & \sigma^{n-1}(c) \\
    \sigma(c) & \sigma(c)^2 & \sigma^3(c) & \dots & c \\
    \vdots & \vdots & \vdots & \ddots & \vdots \\
    \sigma^{n-1}(c) & c & \sigma(c) & \dots & \sigma{n-2}(c)
\end{pmatrix}
\]

sea distinto de cero o, equivalentemente, que \(\{c, \sigma(c), \dots, \sigma^{n-1}(c)\}\) sea una base normal de la extensión de cuerpos \(F^\sigma \subset \F\) \cite{turing1936a}. Recordamos que \(F^\sigma = \F[t]\), donde \(t \in \F\), que puede calcularse como se muestra en \cite{turing1936a}. En nuestro caso particular, el grupo que escogemos es el grupo cíclico \(\{1, \sigma, \dots, \sigma^{n-1}\}\), de manera que el elemento \(t \in \F\) puede obtenerse elegiendo cualquier  (\dots TODO). La existencia de un \(c \in \F\) que genere dicha base está asegurada por el Teorema de la Base Normal. \(\dots\)

\section{Códigos Reed-Solomon Sesgados}%
\label{sec:códigos_reed_solomon_sesgados}

Nuestro siguiente objetivo será proporcionar un método sistemático para contruir CCSs de una determinada distancia Hamming. Debido a la analogía con los códigos Reed-Solomon, los llamaremos \textit{códigos Reed-Solomon sesgados} o \textit{códigos RS sesgados} para abreviar. El siguiente resultado, que es un caso particular de~\cite[Corolario 4.13]{Lam_1988}, será importante en resultados posteriores. Mostraremos una prueba elemental de este.

\begin{lemma}
\label{lem:det_0}
    Sea \(L\) un cuerpo, \(\sigma\) un automorfismo de \(L\) de orden finito \(n\), y \(K = L^\sigma\) el subcuerpo invariante bajo \(\sigma\). Sea  \(\{a_0, \dots, a_{n-1}\}\) una \(K\)-base de \(L\). Entonces, para todo \(t \leq n\), y cada subconjunto \(k_0 < k_1 < \dots < k_{t-1} \subset \{0, 1, \dots, n-1\}\)
    \[
    \begin{vmatrix}
        \alpha_{k_0} & \alpha_{k_1} & \dots & \alpha_{k_{t -1}} \\
        \sigma(\alpha_{k_0}) & \sigma(\alpha_{k_1}) & \dots & \sigma(\alpha_{k_{t-1}}) \\
        \vdots & \vdots & \ddots & \vdots \\
        \sigma^{t-1}(\alpha_{k_0}) & \sigma^{t-1}(\alpha_{k_1}) & \dots & \sigma^{t-1}(\alpha_{k_{t-1}})
    \end{vmatrix}
    \neq 0
    .\]
\end{lemma}

\begin{proof}
    Realizaremos la prueba por inducción sobre \(t\). El caso \(t = 1\) se cumple trivialmente. Por tanto, supongamos que el lema se cumple para un cierto  \(t \geq 1\). Tenemos que comprobar que, para toda matriz \((t+1) \times (t+1)\)
    \[
    \Delta =
    \begin{pmatrix}
        \alpha_{k_0} & \alpha_{k_1} & \dots & \alpha_{k_{t}} \\
        \sigma(\alpha_{k_0}) & \sigma(\alpha_{k_1}) & \dots & \sigma(\alpha_{k_{t}}) \\
        \vdots & \vdots & \ddots & \vdots \\
        \sigma^{t}(\alpha_{k_0}) & \sigma^{t}(\alpha_{k_1}) & \dots & \sigma^{t}(\alpha_{k_{t}})

    \end{pmatrix}
    .\]
el determinante \(|\Delta|\) es distinto de cero. Supongamos por el contrario que \(|\Delta| = 0\). Por la hipótesis de inducción tenemos que las primeras \(t\) columnas de \(\Delta\) son linealmente independientes, luego existen \(a_0, \dots, a_{t-1} \in L\) tales que

\[
(\alpha_{k_t}, \sigma^{t-1}(\alpha_{k_t}), \dots, \sigma^t(\alpha_{k_t})) = \sum_{j=0}^{t-1} a_j(\alpha_{k_j}, \sigma^{t-1}(\alpha_{k_j}), \dots, \sigma^t(\alpha_{k_j}))
.\]

Es decir, \(a_0, \dots, a_{t-1}\) satisfacen el sistema lineal

\begin{equation}
\label{linear_system}
\begin{cases}
    \alpha_{k_t} = a_0\alpha_{k_0} + \cdots + a_{t-1}\alpha_{k_{t-1}} \\
    \sigma(\alpha_{k_t}) = a_0\sigma(\alpha_{k_0}) + \cdots + a_{t-1}\sigma(\alpha_{k_{t-1}}) \\
    \vdots \\
    \sigma^t(\alpha_{k_t}) = a_0\sigma^t(\alpha_{k_0}) + \cdots + a_{t-1}\sigma^t(\alpha_{k_{t-1}})
\end{cases}
.
\end{equation}

Para cada \(j = 0, \dots, t-1\), restamos en (\ref{linear_system}) la ecuación \(j+1\) transformada por \(\sigma^{-1}\) a la ecuación \(j\). Esto produce el siguiente sistema lineal homogéneo

\begin{equation}
\label{hom_liner_system}
\begin{cases}
0 = (a_0 - \sigma^{-1}(a_0))\alpha_{k_0} + \cdots + (a_{t-1} - \sigma^{-1}(a_{t-1}))\alpha_{k_{t-1}} \\
0 = (a_0 - \sigma^{-1}(a_0))\sigma(\alpha_{k_0}) + \cdots + (a_{t-1} - \sigma^{-1}(a_{t-1}))\sigma(\alpha_{k_{t-1}}) \\
\vdots \\
0 = (a_0 - \sigma^{-1}(a_0))\sigma^{t-1}(\alpha_{k_0}) + \cdots + (a_{t-1} - \sigma^{-1}(a_{t-1}))\sigma^{t-1}(\alpha_{k_{t-1}}) \\
\end{cases}
.
\end{equation}

La matriz de coeficientes de (\ref{hom_liner_system}) es no singular, por la hipótesis de inducción, así que tenemos que para todo \(j = 0, \dots, t-1\), \(a_j - \sigma^{-1}(a_j) = 0\), y por tanto \(a_0, \dots, a_{t-1} \in K\). Como consecuencia, la ecuación (\ref{linear_system}) establece una dependencia lineal sobre  \(K\) de la  \(K\)-base  \(\{\alpha_0, \dots, \alpha_{n-1}\}\), creando una contradicción. Por tanto, \(|\Delta| \neq 0\) y el resultado queda demostrado.
\end{proof}

\begin{lemma}
\label{lem:deg_lcm}
    Sea \(\alpha \in \F\) tal que \(\{\alpha, \sigma(\alpha), \dots, \sigma^{n-1}(\alpha)\}\) sea una base de \(\F\) como  \(\F^\sigma\)-espacio vectorial. Fijemos  \(\beta = \alpha^{-1}\sigma(\alpha)\). Para todo subconjunto \(T = \{t_1 < t_2 < \cdots < t_m\} \subset \{0, 1, \dots, n-1\}\), los polinomios
    \[
    g^l = [x - \sigma^{t_1}(\beta), x - \sigma^{t_2}(\beta), \dots, x - \sigma^{t_m}(\beta)]_l
    \]
y
    \[
    g^r = [x - \sigma^{t_1}(\beta^{-1}), x - \sigma^{t_2}(\beta^{-1}), \dots, x - \sigma^{t_m}(\beta^{-1})]_r
    \]
tienen grado \(m\). Además, si \(x - \sigma^s(\beta) |_r g^l\ \) o \(\ x - \sigma^s(\beta^{-1}) |_l g^r\), entonces \(s \in T\).
\end{lemma}

\begin{proof}
    Supongamos que \(\deg g^l < m\), así que \(g^l = \sum_{i=0}^{m-1} g_i x^i\). Como \(g\) es un múltiplo a la izquierda de \(x - \sigma^{t_j}(\beta)\) para todo \(1 \leq j \leq m\), por el lema~\ref{lem:eval} tenemos que

\begin{equation}
\label{linear_system_1}
    \sum_{i=0}^{m-1}g_i N_i(\sigma^{t_j}(\beta)) = 0 \text{ para todo } 1 \leq j \leq m
\end{equation}

Esto es un sistema lineal homogéneo cuya matriz de coeficientes es la traspuesta de la matriz \(M\) dada por
\[
\begin{pmatrix}
    N_0(\sigma^{t_1}(\beta)) & N_0(\sigma^{t_2}(\beta)) & \dots & N_0(\sigma^{t_m}(\beta)) \\
    N_1(\sigma^{t_1}(\beta)) & N_1(\sigma^{t_2}(\beta)) & \dots & N_1(\sigma^{t_m}(\beta)) \\
    \vdots & \vdots & \ddots & \vdots \\
    N_{m-1}(\sigma^{t_1}(\beta)) & N_{m-1}(\sigma^{t_2}(\beta)) & \dots & N_{m-1}(\sigma^{t_m}(\beta))
\end{pmatrix}
.\]

Fijémonos que \(N_i(\sigma^{t_j}(\beta)) = \sigma^{t_j}(N_i(\beta)) = \sigma^{t_j}(N_i(\alpha^{-1}\sigma(\alpha))) = \sigma^{t_j}(\alpha^{-1}) \sigma^{t_{j+i}}(\alpha)\) para todo \(1 \leq j \leq m\) y \(0 \leq i \leq m-1\). Por tanto,  \(|M| = 0\) si y solo si el determinante de la matriz \(M'\),

\[
\begin{pmatrix}
    \sigma^{t_1}(\alpha) & \sigma^{t_2}(\alpha) & \cdots & \sigma^{t_m}(\alpha) \\
    \sigma^{t_1+1}(\alpha) & \sigma^{t_2+1}(\alpha) & \cdots & \sigma^{t_m+1}(\alpha) \\
    \vdots & \vdots & \ddots & \vdots \\
    \sigma^{t_1 + m-1}(\alpha) & \sigma^{t_2+m-1}(\alpha) & \cdots & \sigma^{t_m+m-1}(\alpha)
\end{pmatrix}
,\]

o equivalentemente

\[
\begin{pmatrix}
    \sigma^{t_1}(\alpha) & \sigma^{t_2}(\alpha) & \cdots & \sigma^{t_m}(\alpha) \\
    \sigma(\sigma^{t_1}(\alpha)) & \sigma(\sigma^{t_2}(\alpha)) & \cdots & \sigma(\sigma^{t_m}(\alpha)) \\
    \vdots & \vdots & \ddots & \vdots \\
    \sigma^{m-1}(\sigma^{t_1}(\alpha)) & \sigma^{m-1}(\sigma^{t_2+m-1}(\alpha)) & \cdots & \sigma^{m-1}(\sigma^{t_m}(\alpha))
\end{pmatrix}
,\]
es cero. Sin embargo, por el lema \ref{lem:det_0}, \(|M'| \neq 0\), luego la única solución del sistema lineal (\ref{linear_system_1}) es \(g_0 = \cdots = g_{m-1} = 0\), siendo una contradicción. Por tanto \(\deg g^l = m\). Para el otro polinomio razonamos de forma análoga. Si \(\deg g^r < m\) y \(g^r = \sum_{i=0}^{m-1}g_ix^i\), obtenemos el sistema lineal

\begin{equation}
\label{linear_system_2}
    \sum_{i=0}^{m-1} \sigma^{-i}(g_i)N_{-i}(\sigma^{t_j}(\beta^{-1})) = 0 \text{ para todo } 1 \leq j \leq m.
\end{equation}

Vemos que \(N_{-i}(\sigma^{t_j}(\beta^{-1})) = \sigma^{t_j}(\alpha^{-1})\sigma^{t_j - i + 1}(\alpha)\) para \(0 \leq i \leq m-1\) y \(1 \leq j \leq m\). Entonces, de nuevo por el lema~\ref{lem:det_0} el sistema tiene una única solución \(\sigma^{-i}(g_i) = 0\) para \(0 \leq i \leq m-1\), luego  \(g_0 = g_1 = \cdots = g_{m-1} = 0\). Como dijimos esto es una contradicción y por tanto \(\deg g^r = m\).
\end{proof}

Con esto, ya tenemos los resultados suficientes para definir los códigos sobre los que estarán definidos nuestros algoritmos.

\begin{definition}
\label{def:RS_code}
    Sean \(\alpha, \beta \in \F\) verificando las condiciones del lema \ref{lem:deg_lcm}. Un código Reed-Solomon (RS) sesgado de distancia fijada \(\delta \leq n\) es un CCS generado por \[{[x - \sigma^r(\beta), x - \sigma^{r+1}, \cdots, x - \sigma^{r + \delta -2}(\beta)]}_l,\] para algún \(r \geq 0\).
\end{definition}

\begin{theorem}
\label{th:distance}
    Sea \(\C\) un código RS sesgado de distancia fijada \(\delta\). La distancia Hamming de \(\C\) es \(\delta\).
\end{theorem}

\begin{proof}
Definamos en primer lugar
\[
g = {[x - \sigma^r(\beta), x - \sigma^{r+1}(\beta), \dots, x - \sigma^{r+\delta -2}(\beta)]}_l
\]
un generador de \(\C\) como ideal a la izquierda de \(\mathcal{R}\). Entonces, una matriz de paridad \(H\) de \(\C\) es

\[
\begin{pmatrix}
    N_0(\sigma^r(\beta)) & N_1(\sigma^r(\beta)) & \cdots & N_{n-1}(\sigma^r(\beta)) \\
    N_0(\sigma^{r+1}(\beta)) & N_1(\sigma^{r+1}(\beta)) & \cdots & N_{n-1}(\sigma^{r+1}(\beta)) \\
    \vdots & \vdots & \ddots & \vdots \\
    N_0(\sigma^{r+\delta-2}(\beta)) & N_1(\sigma^{r+\delta-2}(\beta)) & \cdots & N_{n-1}(\sigma^{r+\delta-2}(\beta)) \\
\end{pmatrix}
.\]

pues sus filas dan la evaluación a la derecha de las raices de \(g\). Entonces, por el corolario \ref{cor:distance_parity_matrix}, nos basta con probar no existe ningún conjunto de \(\delta -1\) columnas linealmente dependientes. Para ello procedemos de forma similar a la demostración del lema \ref{lem:deg_lcm}. Como ya utilizamos antes, \(N_i(\sigma^k(\beta)) = \sigma^k(N_i(\beta)) = \sigma^k(\alpha^{-1})\sigma^{i+k}(\alpha)\) para cualesquiera enteros \(i\) y \(k\). Por tanto, dado cualquier conjunto de columnas de tamaño \(\delta -1\), podemos verlo como la matriz M

\[
\begin{pmatrix}
    N_{k_1}(\sigma^r(\beta)) & N_{k_2}(\sigma^r(\beta)) & \cdots & N_{k_{\delta-1}}(\sigma^r(\beta)) \\
    N_{k_1}(\sigma^{r+1}(\beta)) & N_{k_2}(\sigma^{r+1}(\beta)) & \cdots & N_{k_{\delta-1}}(\sigma^{r+1}(\beta)) \\
    \vdots & \vdots & \ddots & \vdots \\
    N_{k_1}(\sigma^{r+\delta-2}(\beta)) & N_{k_2}(\sigma^{r+\delta-2}(\beta)) & \cdots & N_{k_{\delta-1}}(\sigma^{r+\delta-2}(\beta))
\end{pmatrix}
,\]

con \(\{k_1 < k_2 < \cdots < k_{\delta-1} \subset \{0, 1, \dots, n-1\}\). Ahora, \(|M| = 0\), si y solo \(|M'| =0\), donde \(M'\) es la matriz

\[
\begin{pmatrix}
    \sigma^{k_1+r}(\alpha) & \sigma^{k_2+r}(\alpha) & \cdots & \sigma^{k_{\delta-1}+r}(\alpha) \\
    \sigma^{k_1+r+1}(\alpha) & \sigma^{k_2+r+1}(\alpha) & \cdots & \sigma^{k_{\delta-1}+r+1}(\alpha) \\
    \vdots & \vdots & \ddots & \vdots \\
    \sigma^{k_1+r+\delta-2}(\alpha) & \sigma^{k_2+r+\delta-2}(\alpha) & \cdots & \sigma^{k_{\delta-1}+r+\delta-2}(\alpha)
\end{pmatrix}
\]
\[
=
\begin{pmatrix}
    \sigma^{k_1+r}(\alpha) & \sigma^{k_2+r}(\alpha) & \cdots & \sigma^{k_{\delta-1}+r}(\alpha) \\
    \sigma(\sigma^{k_1+r}(\alpha)) & \sigma(\sigma^{k_2+r})(\alpha) & \cdots & \sigma(\sigma^{k_{\delta-1}+r}(\alpha)) \\
    \vdots & \vdots & \ddots & \vdots \\
    \sigma^{\delta-2}(\sigma^{k_1+r}(\alpha)) & \sigma^{\delta-2}(\sigma^{k_2+r}(\alpha)) & \cdots & \sigma^{\delta-2}(\sigma^{k_{\delta-1}+r}(\alpha))
\end{pmatrix}
.\]

Por ser \(\{\alpha, \sigma(\alpha), \dots, \sigma^{n-1}(\alpha)\) una base de la extensión \(\F^\sigma \subset \F\), por el lema \ref{lem:det_0}, \(|M'| \neq 0\), y por tanto esas columnas son linealmente independientes.
\end{proof}

\section{Algoritmo Principal}%
\label{sec:algoritmo_principal}

De aquí en adelante \(\C\) denotará un código RS sesgado de distancia fijada \(\delta\) generado, como ideal a la izquierda de  \(\mathcal{R}\), por  \(g = {[x - \sigma^r(\beta), x - \sigma^{r+1}(\beta), \dots, x - \sigma^{r+\delta - 2}(\beta)]}_l\), para algún \(r \geq 0 \), donde \(\beta\) lo elegimos como en la definición~\ref{def:RS_code}. Sabemos que la distancia mínima de \(\C\) es exactamente \(\delta\) por el teorema~\ref{th:distance}. Sea \(\tau = \lfloor (\sigma -1)/2 \rfloor\) que es el máximo número de errores que nuestro código puede corregir. Por simplicidad, supondremos que \(r = 0\). Esto no es una restricción, pues siempre podemos escribir  \(\beta' = \sigma^r(\beta)\). Entonces, \(\beta' = {(\alpha')}^{-1} \sigma(\alpha')\), donde \(\alpha' = \sigma^r(\alpha)\), y es claro que \(\alpha'\) también proporciona una base normal. Por tanto, \(g = [x - \beta', x - \sigma(\beta'), \dots, x - \sigma^{\delta - 2}(\beta')]\).

Sea \(c \in \C\) una palabra que es transmitida a través de un canal ruidoso y el polinomio \(y = c + e\) es recibido, donde \(e = e_1 x^k + \cdots + e_\nu x^{k_\nu}\) con \(\nu \leq \tau\). Definimos el polinomio \textit{localizador de errores} como
\[
\lambda = {\left[1 - \sigma^{k_1}(\beta)x, 1 - \sigma^{k_2}(\beta)x, \dots, 1 - \sigma^{k_\nu}(\beta)x\right]}_r
.\]

En primer lugar mostraremos que \(\lambda\) determina las posiciones con un error no nulo.

\begin{lemma}\label{lem:lamda_roots}
    Para cualquier subconjunto \(\{t_1, \dots, t_m\} \subset \{0, 1, \dots, n-1\}\) ,
    \[
    {\left[1 -\sigma^{t_1}(\beta)x, \dots, 1 - \sigma^{t_m}(\beta)x \right]}_r
    = {\left[x - \sigma^{t_1 - 1}(\beta^{-1}), \dots, x - \sigma^{t_m -1}(\beta^{-1})\right]}_r
    .\]
\end{lemma}

\begin{proof}
    Para cualquier \(a \in \F\),
    \[
    1 - ax = (x - \sigma^{-1}(a^{-1}))(- \sigma^{-1}(a)),\quad
    x - \sigma^{-1}(a^{-1}) = (1 - ax)(-\sigma^{-1}(a^{-1})).
    \]
    Entonces, los polinomios del resultado se dividen a la izquierda mutuamente, y por tanto ambos mínimos comunes múltiplos coinciden.
\end{proof}

\begin{proposition}
\label{prop:root_error_position}
    \(1- \sigma^d(\beta)x\) divide a la izquierda a \(\lambda\) si y solo si \(x - \sigma^{d-1}(\beta^{-1})\) divide a la izquierda a  \(\lambda\) si y solo si \(d \in \{k_1, \dots, k_\nu\}\).
\end{proposition}

\begin{proof}
    Por el lema anterior, \(1 - \sigma^d(\beta)x\) divide a la izquierda \(\lambda\) si y solo si  \(x - \sigma^{d-1}(\beta^{-1})\) divide a la izquierda a \(\lambda\). Ahora, por el lema \ref{lem:deg_lcm}, \(x - \sigma^{d-1}(\beta^{-1})\) es un divisor a la izquierda de \(\lambda\) si y solo si  \(d \in \{k_1, \dots, k_\nu\}\).
\end{proof}

Por tanto, una vez que \(\lambda\) es conocido, las coordenadas del error pueden ser localizadas siguiendo la siguiente regla: \(d \in {0,1, \dots, n-1}\) es una posición de error si y solo si \(\sigma^{d-1}(\beta{-1})\) es una raiz a la izquierda de \(\lambda\). En realidad,  \(\lambda\) puede ser sustituido por cualquier otro polinomio en \(R\) asociado a la derecha a \(\lambda\), es decir, cualquier polinomio que difiera de \(\lambda\) en la multiplicación a la derecha por un elemento no nulo de \(\F\).

Para cualquier \(1 \leq j \leq \nu\), \(\lambda = (1 - \sigma^{k_j}(\beta)x)p_j\) para algún \(p_j \in R\) con \(\deg p_j = \nu - 1\). Definimos entonces el polinomio \textit{evaluador de errores} como \(\omega = \sum_{j=1}^\nu e_j \sigma^{k_j}(\alpha)p_j\). Así, si conocemos el polinimio localizador de errores \(\lambda\) y el polinomio evaluador de errores \(\omega\), podremos calcular los valores \(e_1, e_2, \ldots, e_{\nu}\) resolviendo un sistema lineal dado por \(\omega = \sum_{j=1}^{\nu} e_j \sigma^{k_j}(\alpha)p_j\). Además, es directo comprobar que \(\deg \omega < \nu\), pues, como ya dijimos, \(\deg(p_j) = \nu -1\) para todo \(1 \le j \le \nu\).

Finalmente, para cada \(0 \leq i \leq n-1\), el \(i\)-ésimo síndrome \(S_i\) del polinomio recibido  \(y = \sum_{j=0}^{n-1} y_j x^j\) se define como el resto de la división a la derecha de  \(y\) por  \(x - \sigma^i(\beta)\). Este \(S_i\) es la evaluación a la derecha de \(y\) en  \(\sigma^i(\beta)\). Siempre que \(0 \leq i \leq 2\tau -1 = \delta -2\), las evaluaciónes a la derecha en \(c\) son cero, y por tanto
 \[
\begin{aligned}
    S_i &= \sum_{j=0}^{n-1} y_j N-j(\sigma^i(\beta)) \\
    &= \sum_{j=1}^\nu e_j N_{k_j}(\sigma^i(\beta))\\
    &= \sum_{j=1}^\nu e_j \sigma^i(N_{k_j}(\beta))\\
    &= \sum_{j=1}^\nu e_j \sigma^i(\alpha^{-1})\sigma^{k_j + i}(\alpha)\\
    &= \sigma^i(\alpha^{-1})\sum_{j=1}^\nu e_j \sigma^{k_j + i}(\alpha)
\end{aligned}
.\]

Por esto, \(\sigma^i(\alpha)S_i = \sum_{j=1}^\nu e_j \sigma^{k_j + i}(\alpha)\) y llamamos al polinomio \(S = \sum_{i=0}^{2\tau -1} \sigma^i(\alpha) S_i x^i\) el polinomio síndrome de \(y\).

\begin{theorem}
    Los polinomios localizador de errores y evaluador de errores cumplen la ecuación clave no conmutativa:
    \[
    \omega = S\lambda + x^{2\tau}u.
    \]
    donde \(u \in R\) es de grado menor que \(\nu\).
\end{theorem}

\begin{proof}
Por definición sabemos que
\[
S = \sum_{i = 0}^{2\tau-1} \sum_{j=1}^{\nu} e_j \sigma^{k_j + i}(\alpha) x^i
,\]
y que para cualquier \(1 \leq j \leq \nu\)
\[
\lambda = (1 - \sigma^{k_j}(\beta)x)p_j
.\]
Queremos llegar a que
\[
\omega = \sum_{j=1}^{\nu} e_j \sigma^{k_j}(\alpha)p_j = S\lambda + x^{2\tau u}
\]
para algún \(u\) de grado menor que \(\nu\). Tomamos ahora \(u = \sum_{i = 0}^{2\tau-1} \sum_{j=1}^{\nu} \sigma^{-2\tau}(e_j)\sigma^{k_j}(\alpha)p_j\). Entonces
\[
\begin{aligned}
S\lambda + x^{2\tau} u &= \sum_{i = 0}^{2\tau-1} \sum_{j=1}^{\nu} e_j\sigma^{k_j + i}(\alpha)x^i(1-\sigma^{k_j}(\beta)x)p_j + x^{2\tau}\sigma^{-2\tau}(e_j)\sigma^{k_j}(\alpha)p_j \\
&= \sum_{j=1}^{\nu} e_j \left( \sum_{i = 0}^{2\tau-1} \sigma^{k_j + i}(\alpha)x^i(1-\sigma^{k_j}(\beta)x) + x^{2\tau}\sigma^{k_j}(\alpha)\right)p_j
\end{aligned}
,\]
tras intercambiar las sumatorias, y sacar factor común \(e_j\) y \(p_j\). Ahora nos centramos en la sumatoria interior.

\[
\begin{aligned}
&\sum_{i = 0}^{2\tau-1} \sigma^{k_j + i}(\alpha)x^i(1-\sigma^{k_j}(\beta)x) + x^{2\tau}\sigma^{k_j}(\alpha) \\
&= \sum_{i = 0}^{2\tau-1} x^i \sigma^{k_j}(\alpha) (1-\sigma^{k_j}(\beta)x) + x^{2\tau}\sigma^{k_j}(\alpha) \\
&= \sum_{i = 0}^{2\tau-1} x^i \sigma^{k_j}(\alpha) - x^i \sigma^{k_j}(\alpha)\sigma^{k_j}(\alpha^{-1})\sigma^{k_j + 1}(\alpha) x) + x^{2\tau}\sigma^{k_j}(\alpha)\\
&=\sum_{i = 0}^{2\tau-1} x^i \sigma^{k_j}(\alpha) - x^{i+1} \sigma^{k_j}(\alpha) + x^{2\tau}\sigma^{k_j}(\alpha)\\
&=\left(\sum_{i = 0}^{2\tau-1} x^i  - x^{i+1} + x^{2\tau}\right)\sigma^{k_j}(\alpha) = \sigma^{k_j}(\alpha)
\end{aligned}
.\]
Con esto, sustituyendo en la expresión anterior tenemos que
\[
\begin{aligned}
S\lambda + x^{2\tau} u &= \sum_{j=1}^{\nu} e_j \left( \sum_{i = 0}^{2\tau-1} \sigma^{k_j + i}(\alpha)x^i(1-\sigma^{k_j}(\beta)x) + x^{2\tau}\sigma^{k_j}(\alpha)\right)p_j \\
&= \sum_{j=1}^{\nu} e_j \sigma^{k_j}(\alpha) p_j = \omega
\end{aligned}
.\]
\end{proof}

Ahora intentaremos resolver esta ecuación, para lo que utilizaremos el Algoritmo Euclídeo Extendido a la derecha presentado en \ref{alg:right_ext_euc_alg}. Recordemos que, para cualesquiera $f,g \in R$, cada paso $i$ del algoritmo proporciona coeficientes $\{u_{i}, v_{i}, r_{i}\}$ tales que $fu_{i} + fv_{i} = r_{i}$, donde ${(f,g)}_l = r_h$ y $\deg r_{i+1} < \deg r_{i}$ para cualquier $0 \le i \le h-1$.

\begin{theorem}
\label{th:mult_key_eq}
    La ecuación clave no conmutativa
    \begin{equation}
    \label{eq:key_equation}
        x^{2\tau}u + S\lambda = \omega
    \end{equation}
    es un múltiplo a la derecha de la ecuación
    \begin{equation}
    \label{eq:key_equation_div}
        x^{2\tau}u_{I} + Sv_{I} = r_{I},
    \end{equation}
    donde $u_{I}, v_{I}$ y $r_{I}$ son los coeficientes de Bezout dados por el Algoritmo Euclídeo Extendido a la derecha con entrada $x^{2\tau}$ y \(S\), e \(I\) es el índice determinado por las condiciones \(\deg r_{I-1} \ge \tau\) y \(\deg r_{I} < \tau\). En particular, \(\lambda = v_{I}g\) y \(\omega = r_{I}g\) para algún \(g \in R\).
\end{theorem}

\begin{proof}
    Recordemos que \(\deg S < 2\tau\), \(\deg \lambda \le \nu \le \tau\) y que \(\deg \omega < \nu \le \tau\). Entonces, \(\deg u < \tau\), pues en otro caso \(\deg x^{2\tau}u \ge 3\tau > \deg S\lambda\), y por tanto \(\deg \omega \ge 3\lambda\) que es contradicción. Por otro lado, por el lema \ref{lem:reea} \textit{VI)}, \(\deg v_{I} + \deg r_{I-1} = 2\tau\), y utilizando la hipótesis \(\deg r_{I-1} \ge \tau\) tenemos que \(\deg v_{I} \le \tau\).

    Consideremos ahora el mínimo común múltiplo a la derecha \({[\lambda, v_{I}]}_{r} = \lambda a = v_{I}b\), donde \(a, b \in R\) con \(\deg a \le \deg v_{I} \le \tau\) y \(\deg b \le \deg \lambda \le \tau\). Entonces  \({(a,b)}_{r} = 1\). Entonces, multiplicando \ref{eq:key_equation} a la derecha por \(a\), y \ref{eq:key_equation_div} a la derecha por \(b\), obtenemos
    \begin{equation}
        x^{2\tau} u a + S \lambda a = \omega a
    \end{equation}
y
    \begin{equation}
        x^{2\tau}u_{I}b + Sv_{I}b = r_{I}b.
    \end{equation}

    Ahora, restando ambas ecuaciones obtenemos \(x^{2\tau}(ua - u_{I}b) = \omega a - r_{I}b\). Por \(\deg \omega < \tau, \deg a \le \tau, \deg r_{I} < \tau, \deg{b} \le \tau\), tenemos que \(\deg (\omega a - r_{I}b) < 2\tau\), luego \(ua = u_{I}b\) y \(\omega a = r_{I}b\), pues en caso contrario el grado del término izquierdo sería mayor o igual que \(2\tau\). De hecho \({(a,b)}_{r} = 1\) nos lleva a que \({[u, u_{I}]}_r = ua = u_{I}b\) y \({[\omega, r_{I}]}_r = \omega a = r_{I}b\), y en particular \(\deg a \le \deg r_{I} < \tau\).

    Sea \({[a,b]}_l = a'a = b'b\). Por ser \([\lambda, v_{I}]_r\) un múltiplo a la izquierda de \(a\) y \(b\), existe \(m \in R\) tal que \({[\lambda, v_{I}]}_{r} = m{[a,b]}_{l}\), es decir, \(\lambda a = v_{I}b = ma'a = mb'b\). Por esto, \(\lambda = ma'\) y \(v_{I} = mb'\) y, por minimalidad, \({(\lambda, v_{I})}_l = m\). Podemos utilizar argumentos similares para probar que existen \(m', m'' \in R\) tales que \(u_{I} = m'b'\) y  \(u = m'a'\), y  \(r_{I} = m''b'\) y \(\omega = m''a'\). Por el lema \ref{lem:reea} \textit{V)} \({(u_{I}, v_{I})} _r = 1\), luego \(b' = 1\). Esto completa la prueba, pues tenemos \(b = b b' = a a'\), y por tanto  \(\lambda = v_{I}a'\), \(\omega = r_{I}a'\) y \(u = u_{I}a'\).
\end{proof}

Usando la notación del teorema recién demostrado, vemos que, por \(\lambda = v_{I}a'\) y \(\omega = r_{I}a'\), si \({{(\lambda, \omega)}_{r} = 1}\) entonces \(\lambda = v_{I}\) y \(\omega = r_{I}\). En este caso, el teorema \ref{th:mult_key_eq} proporciona un método algorítmico para calcular ambos polinomios, el localizador de errores y el evaluador de errores. Como ya dijimos, estos dos polinomios nos permiten calcular el error al recibir un polinomio \(y = c + e\), con \(c\) y \(e\) cumpliendo las hipótesis previamente mencionadas, por tanto, podemos describir un algoritmo de decodificación para códigos RS sesgados (vease el algoritmo~\ref{alg:decoding_1}) que será válido siempre que \({(\lambda, \omega)}_{r} = 1\).

\begin{algorithm}[H]
 \label{alg:decoding_1}
 \SetKwInput{KwIn}{Entrada}
 \SetKwInput{KwOut}{Salida}
 \SetKw{Initialization}{Inicialización:}
 \KwIn{Un polinomio \(y = \sum_{i=0}^{n-1}y_i x^{i}\) obtenido de la transmisión de una palabra código \(c\) perteneciente a un código RS sesgado \(C\) generado por \(g = {[\{x-\sigma^{i}(\beta)\}_{i=0,\ldots,\delta-2}]}_{l}\) con capacidad para corregir \(\tau = \lfloor (\delta -1)/2 \rfloor\) errores.}
 \KwOut{Una palabra código \(c'\) o un \textit{error de ecuación clave}.}
 \For{\(0 \le i \le 2\tau -1\)}{
    \(S_i \gets \sum_{j=0}^{n-1} y_j N_{j}(\sigma^{i}(\beta))\)
 }
 \(S \gets \sum_{i=0}^{2\tau -1} \sigma^{i}(\alpha)S_{i}x^{i}\)\\
 \If{\(S = 0\)}{
    \Return{\(y\)}
 }
 \({\{u_{i}, v_{i}, r_{i}\}}_{i=0,\ldots,l} \gets \text{REEA}(x^{2\tau}, S)\)\\
 \(I \gets \text{ primera iteración en REEA tal que } \deg r_{i} < \tau\)\\
 \(pos \gets \emptyset\)\\
 \For{\(0 \le i \le n-1\)}{
     \If{ \(\sigma^{i-1}(\beta^{-1}) \text{ es una raiz a la izquierda de } v_{I}\) }{
        \(pos = pos \cup \{i\}\)
     }
 }
\If{\(\deg v_{I} > \operatorname{Cardinal}(pos)\) }{
    \Return{ error en la ecuación clave }
}
\For{ \(j \in pos\) }{
    \(p_j \gets \operatorname{rquot}(v_{I}, 1 - \sigma^{j}(\beta)x)\)
}

Resolver el sistema lineal \(r_I = \sum_{j\in pos} e_{j}\sigma^{j}(\alpha)p_{j}\)\\
\(e \gets \sum_{j\in pos}e_{j}x^{j}\)\\
\Return{\(y - e\)}
 \caption{Algoritmo de decodificación para códigos RS sesgados}
\end{algorithm}

\section{Errores en la ecuación clave}%
\label{sec:errores_en_ecuación_clave}

En esta sección abordaremos el problema producido por un error en la ecuación clave, es decir, cuando el máximo común divisor a la derecha de los polinomios localizador y evaluador de errores no coincide con la unidad. Nuestro objetivo entonces es encontrar un método que nos permita, aun existiendo el error mencionado, encontrar el polinomio localizador de errores.

A lo largo de toda esta sección seguiremos la notación utilizada en la sección~\ref{sec:algoritmo_principal}.

El primer resultado que mostramos será que, si se produce un único error en el mensaje, en cuyo caso el polinomio localizador será de grado uno, este siempre puede ser corregido.

\begin{lemma}
    \(\deg v_I \ge 1\). Como consecuencia, si \(\deg \lambda = 1\), entonces  \(v_I\) y  \(\lambda\) son asociados a la derecha.
\end{lemma}

\begin{proof}
    La prueba se basa en los grados de los polinomios del teorema~\ref{th:mult_key_eq}. Por este teorema, \(\omega = r_I g\) para algún \(g \in R\). Como ya mencionamos anteriormente, \(\deg \omega < \nu\), y por tanto \(\deg g < \nu\). Ahora, el mismo teorema nos dice que \(\lambda = v_I g\), y por el teorema~\ref{lem:deg_lcm} tenemos que \(\nu = \deg \lambda = \deg v_{I} + \deg g\). De aquí sigue que \(\deg v_I \ge 1\).
\end{proof}

Volvamos ahora a considerar el caso en el que se produce un error en la ecuación clave, y como afrontarlo. Por el teorema~\ref{th:mult_key_eq} y siguiendo el algoritmo~\ref{alg:decoding_1} obtenemos los polinomios \(r_I, u_i, v_I\) que cumplen la ecuación \(x^{2\tau}u_{I} + Sv_{I} = r_{I}\), con \(\lambda = v_I g\) y  \(\omega = r_I g\) para algún \(g \in R\).

Ahora bien, por el lema recién demostrado, sabemos que \(\deg v_I \ge 1\). Si \(\deg g = 0\), entonces \(v_I\) nos sirve como polinomio localizador, y el algoritmo~\ref{alg:decoding_1} decodificará correctamente el polinomio recibido. En caso contrario, nuestra estrategia consistirá en, empezando por \(v_I\), encontrar una cadena creciente de divisores a la izquierda de \(\lambda\), lo que nos permitirá encontrar una posición nueva del error por cada elemento de dicha cadena. En primer lugar, queremos demostrar un criterio que nos diga si hemos encontrado ya todas las posiciones del error, y de ser así tendremos el polinomio localizador. Para demostrar esto necesitaremos utilizar el siguiente lema.

\begin{lemma}
\label{lem:direct-sum}
    Sean \(\{t_1 < t_2 < \cdots < t_m \} \subset \{0, 1, \dots, n-1 \}\) con \(m > 1\), y
     \[
    q = {[1 - \sigma^{t_1}(\beta)x, 1 - \sigma^{t_2}(\beta)x, \ldots, 1 - \sigma^{t_m}(\beta)x]}_r
    .\]
    Sean \(q_1, q_2, \ldots, q_m\) tales que \(q = (1 - \sigma^{t_j}(\beta)x)q_j\) para cualquier \(1 \le j \le m\). Entonces:
\begin{nlist}
    \item \({[q_1, q_2, \ldots, q_m]}_l = q\) y \({(q_1, q_2, \ldots, q_m)}_r = 1\).
    \item \(R/Rq = \bigoplus_{j = 1}^{m} Rq_j/Rq\).
    \item El conjunto \(\{q_1, \ldots, q_m\}\) módulo \(Rq\) proporciona una base de  \(R/Rq\) como un \(\F\)-espacio vectorial.
\end{nlist}
\end{lemma}

\begin{proof}
\begin{nlist}
    \item Utilizando el lema~\ref{lem:lamda_roots} para seguidamente poder aplicar el lema~\ref{lem:deg_lcm} tenemos que \(\deg q = m\), y por tanto
    \[
    \deg q_j = m-1 \quad \forall j \in \{1, \ldots, m\}.
    \]

    Como \(m > 1\), el grado del mínimo común múltiplo a la derecha de cualesquiera dos elementos en \(\{q_1, q_2, \ldots, q_m\}\) debe ser al menos \(m - 1 + 1\), luego el grado de  \({[q_1, \ldots, q_m]}_l\) debe ser al menos \(m\). Pero \(q\) es claramente un mínimo común múltiplo a la izquierda de  \(q_1, \ldots, q_m\). Así, \(q = {[q_1, \ldots, q_m]}_l \).

Veamos ahora que \(p = {(q_1, q_2, \ldots, q_m)}_r = 1\). Supongamos \(p \neq 1\), y  \(q_j = q_j'p\) para todo \(j = 1, \ldots, m\). Entonces
\[
q = (1 - \sigma^{t_j}(\beta)x)q_j'p \quad \forall j \in \{1, \ldots, m\}
.\]
Esto es una contradicción pues, si decimos
\[
q' = (1 - \sigma^{t_1}(\beta)x)q_1' = \cdots = (1 - \sigma^{t_m}(\beta)x)q_m'
,\]
entonces \(q'\) es un múltiplo común a la derecha de cada \(1 - \sigma^{t_j}(\beta)x\) de grado menor que \(q\) así que este no podría ser el mínimo común múltiplo a la derecha de dichos términos.
    \item Cada uno de los polinomios \(q_j\) es un divisor a la derecha de \(q\) y por tanto \(Rq \subset Rq_j\) para \(j = 1, \ldots, m\). Por esto, como \({(q_1, \ldots, q_m)}_r = 1\) tenemos que
    \[
    R/Rq = \sum_{j=1}^{m} Rq_j/Rq
    .\]
Además, por \(q = (1 - \sigma^{t_j}(\beta)x)q_j\) para \(1 \le j \le m\),  \(Rq_j / Rq \equiv R / R(1-\sigma^{t_j}(\beta)x)\) es uno dimensional sobre \(F\). Así, como la dimensión de \(R / Rq\) como  \(\F\)-espacio vectorial es  \(\deg q = m\), tenemos la suma directa que buscábamos.

    \item Se obtiene directamente del apartado (II)
    \item Se obtiene directamente del apartado (II)
\end{nlist}
\end{proof}

\begin{theorem}
    Sean \(q, p, s \in R\)  tales que \(x^{2\tau}q + Sp = s\),  \(qg = u, pg = \lambda, \text{ y  } sg = \omega\) para algún \(g \in R\). Consideremos \(T = \{t_1, t_2, \ldots, t_{m}\} \subset \{0,1, \ldots, n-1\}\) el conjunto de índices que verifican que \(\sigma^{j-1}(\beta^{-1})\) es una raiz a la izquierda de \(p\) si y solo si  \(j \in T\). Entonces \(m = \deg p \) si y solo \(g\) es una constante.
\end{theorem}

\begin{proof}
    En primer lugar mencionemos que los elementos de \(T\) son en realidad posiciones del error. Esto se cumple ya que dividen por la izquiera a p, y por \(\lambda = pg\), también a \(\lambda\), así que podemos aplicar la proposición~\ref{prop:root_error_position}. Dicho esto, reordenamos el conjunto de las posiciones del error de forma que \(T = \{k_1, k_2, \ldots, k_m\}\).

    Si \(\deg g = 0\), entonces las raices de \(p\) coinciden con las de  \(\lambda\), y por tanto, por el lema~\ref{lem:deg_lcm}, \(m = \deg p = \deg \lambda = \nu\). Supongamos ahora que \(m = \deg p\). Por el lema~\ref{lem:deg_lcm} sabemos que \( [x - \sigma^{k_1}(\beta^{-1}), \ldots, x - \sigma^{k_m}(\beta^{-1})]_r \) tiene grado \(m\), y por tanto tiene el mismo grado que \(p\). Por esto
\(
p = [x - \sigma^{k_1-1}(\beta^{-1}), x - \sigma^{k_2-1}(\beta^{-1}), \dots, x - \sigma^{k_m-1}(\beta^{-1})]_r
\)
, y aplicando directamente el lema~\ref{lem:lamda_roots} tenemos que:
    \[
    p = {[1 - \sigma^{k_1}(\beta)x, \ldots, 1 - \sigma^{k_m}(\beta)x]}_{r}
    .\]

Por el lema~\ref{lem:direct-sum} (III), cada polinomio de grado menor que \(m\) puede escribirse como una combinación  \(\F\)-lineal de los polinomios  \(p_1', \ldots, p_m'\). Ahora, por \(sg = \omega\) y \(pg = \lambda\),
\[
\deg s = \deg \omega - \deg g = \deg \omega + \deg p - \deg \lambda \le \nu -1 + m - \nu = m-1
,\]
luego podemos escribir \(s\) de la forma \(\sum_{i=1}^{m} a_ip_i'\) para ciertos \(a_1, a_2, \ldots, a_m \in F\). Recordemos que, siguiendo la notación de la sección anterior, los polinomios \(p_1, \ldots, p_\nu\) eran aquellos tales que \(\lambda = (1 - \sigma^{k_j}(\beta)x)p_j\) donde \(k_j\) era una posición del error cualquiera. Entonces, por \(\lambda = pg\), para cada \(j = 1, \ldots, m\) se cumple que \((1 - \sigma ^{k_j}(\beta)x)p_j = (1 - \sigma^{k_j}(\beta)x)p_j'g\), luego \(p_j = p_j'g\). Ahora, usando que \(sg = \omega\), tenemos

\begin{equation}
\label{eq:m_nu}
\begin{aligned}
    \omega &= \sum_{j=1}^{m}e_{j}\sigma^{k_j}(\alpha)p_j + \sum_{j=m+1}^{\nu}e_{j}\sigma^{k_j}(\alpha)p_j = sg  = \left( \sum_{j=1}^{m}a_j p_j' \right) g \\
    &= \sum_{j=1}^{m} a_j p_j
\end{aligned}
.
\end{equation}

Ahora, volviendo a aplicar el lema~\ref{lem:direct-sum} (IV), pero esta vez sobre todas las posiciones de error y el polinomio localizador, obtenemos que \(\{p_1, \ldots, p_{\nu}\}\) es una base de \(R / R\lambda\) como  \(\F\)-espacio vectorial. Así, como \(e_i\sigma^{k_i} \neq 0\) para cada \(i \le \nu\), la ecuación~\ref{eq:m_nu} nos dice que \(m = \nu\), y por tanto, \(\deg g = 0\).
\end{proof}

Como ya dijimos, lo recién demostrado nos será de utilidad para saber si hemos encontrado todas las posiciones del error. Por tanto, el siguiente paso será construir el algoritmo que nos permita, en caso de existir, encontrar una posición nueva del error. De esta forma, en caso de que se produzca un error en la ecuación clave en el algoritmo~\ref{alg:decoding_1}, podremos hallar nuevas posiciones del error hasta que sepamos que las tenemos todas.

\begin{algorithm}[H]
 \label{alg:find_a_position}
 \SetKwInput{KwIn}{Entrada}
 \SetKwInput{KwOut}{Salida}
 \SetKw{Initialization}{Inicialización:}
 \KwIn{Un polinomio no constante \(p\) tal que \(\lambda = pg\) para algún \(g \in R\), \(pos = \{i \ge 0 \text{ tales que } (1 - \sigma^{i}(\beta)x) {|}_l p \}\), con \(\deg p > \operatorname{Cardinal}(pos)\).}
 \KwOut{\(d \not \in pos\) tal que \((1 - \sigma^{d}(\beta)x)\) divide a la izquierda a \(\lamda\).}
 \(f \gets p\),  \(e \gets \deg f\) \\
 \For{\(0 \le i \le n-1\)}{
     \If{\(i \not \in pos\)}{
        \(f \gets {[f, 1 - \sigma^{i}(\beta)x]}_r\) \\
        \If{\(\deg f = e\)}{
            \Return{i}
        }
        \Else{
            \(e \gets e + 1\)
        }
     }
 }
 \caption{Encuentra-una-posición}
\end{algorithm}

\begin{proposition}
    El algoritmo~\ref{alg:find_a_position} encuentra una nueva posición del error correctamente.
\end{proposition}
\begin{proof}
    Sea \(T = \{t_1, \ldots, t_r\} = \{0, 1, ..., n-1\} \setminus pos\). Definimos ahora
\[
\begin{aligned}
&\lambda_0 = \lambda, \quad \lambda_i = {[\lambda _{i-1}, 1 - \sigma^{t_i}(\beta)x]}_r\quad \text{para } 1 \le i \le r \\
&f_0 = p, \quad f_i = {[f _{i-1}, 1 - \sigma^{t_i}(\beta)x]}_r\quad \text{para } 1 \le i \le r
\end{aligned}
.\]

Veamos que \(f_i {|}_l \lambda_i\) para cualquier \(i = 0, \ldots, r\). En efecto, para \(i = 0\) el resultado es cierto por hipótesis, y supuesto que se cumple para \(i-1\), \(f_{i-1}\) divide a la izquierda a \(\lambda_i\) por dividir a \(\lambda_{i-1}\), y \(1- \sigma^{t_i}(\beta)x\) lo divide también de forma trivial, luego por definición el mínimo común múltiplo a la derecha de estos divide a \(\lambda_i\). Dicho esto, probaremos en primer lugar que el algoritmo devuelve una posición del error.

Supongamos que la secuencia \(\{\deg f_i\}_{0 \le i \le r}\) es estrictamente creciente. Por esto, \(\deg f_r = r + \deg p > n - \operatorname{Cardinal}(pos) + \deg p > n\). (TODO No entiendo) Esto sin embargo no es posible, pues \(f_r\ {|}_l\ \lambda_r = x^{n} - 1\). Por tanto, existe un \(d \le 0\) mínimo tal que \(\deg f_{d-1} = \deg f_{d}\). Entonces, \(1 - \sigma^{t_d}(\beta)x \ {|}_l\ f_{d-1}\ {|}_l\ \lambda_{d-1} = {[\lambda, 1 - \sigma^{t_1}(\beta)x, \ldots, 1 - \sigma^{t_{d-1}}(\beta)x]}_r \). Como \(t_{d} \neq t_1, \ldots, t_{d-1}\),  \(1 - \sigma^{t_d}(\beta)x \) divide a la izquierda \(\lambda\) concluyendo la prueba.
\end{proof}
