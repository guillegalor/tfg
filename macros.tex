%------------------------
% Bibliotecas para matemáticas de latex
%------------------------
\usepackage{amsthm}
\usepackage{amsmath}
\usepackage{tikz}
\usepackage{tikz-cd}
\usetikzlibrary{shapes,fit}
\usepackage{bussproofs}
\EnableBpAbbreviations{}
\usepackage{mathtools}
\usepackage{scalerel}
\usepackage{stmaryrd}

%------------------------
% Estilos para los teoremas
%------------------------
\theoremstyle{plain}
\newtheorem{theorem}{Teorema}
\newtheorem{proposition}{Proposición}
\newtheorem{lemma}{Lema}
\newtheorem{corollary}{Corolario}
\theoremstyle{definition}
\newtheorem{definition}{Definición}
\newtheorem{proofs}{Demostración}
\theoremstyle{remark}
\newtheorem{remark}{Comentario}
\newtheorem{exampleth}{Ejemplo}

\begingroup\makeatletter\@for\theoremstyle:=definition,remark,plain\do{\expandafter\g@addto@macro\csname th@\theoremstyle\endcsname{\addtolength\thm@preskip\parskip}}\endgroup

%------------------------
% Macros
% ------------------------

% Aquí pueden añadirse abreviaturas para comandos de latex
% frequentemente usados.
\newcommand*\diff{\mathop{}\!\mathrm{d}}

% Comandos tomados de la plantilla de libreim

% Redefinir letra griega épsilon.
\let\epsilon\upvarepsilon

% Valor absoluto: \abs{}
\providecommand{\abs}[1]{\lvert#1\rvert}

% Fracción grande: \ddfrac{}{}
\newcommand\ddfrac[2]{\frac{\displaystyle #1}{\displaystyle #2}}

% Texto en negrita en modo matemática: \bm{}
\newcommand{\bm}[1]{\boldsymbol{#1}}

% Letras de conjuntos
\newcommand{\R}{\mathbb{R}}
\newcommand{\N}{\mathbb{N}}
\newcommand{\F}{\mathbb{F}}
\newcommand{\C}{\mathcal{C}}

% Sucesiones
\newcommand{\xn}{\{x_n\}}
\newcommand{\fn}{\{f_n\}}

% Letra griega "chi" en línea con el texto
\DeclareRobustCommand{\rchi}{{\Large \mathpalette\irchi\relax}}
\newcommand{\irchi}[2]{\raisebox{0.4\depth}{$#1\chi$}} % inner command, used by \rchi

% Letra 'omega'
\newcommand{\W}{\Omega}
\newcommand{\w}{\omega}

% Conjunto vacío
\renewcommand{\O}{\emptyset}

% Coeficiente Lider
\newcommand{\lc}{\operatorname{lc}}

% Listas ordenadas con números romanos (i), (ii), etc.
\newenvironment{nlist}
{\begin{enumerate}
    \renewcommand\labelenumi{(\emph{\roman{enumi})}}}
  {\end{enumerate}}
