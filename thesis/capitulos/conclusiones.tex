\chapter*{Conclusiones}%
\label{chap:conclusiones}

Como dijimos en la introducción de este trabajo, los objetivos de este eran estudiar los fundamentos de los códigos lineales, las extensiones de Ore, y la versión del algoritmo de Sugiyama para la familia de códigos skew RS, así como llevar a cabo la implementación del algoritmo y la representación de estos códigos en el entorno SageMath.

Podemos concluir que todos estos objetivos han sido alcanzados, pues a lo largo de los tres primeros capítulos hemos realizado el estudio de todo lo que propusimos, y hemos realizado la implementación prometida, cuya documentación se encuentra por completo en el último capítulo.

Como futuro trabajo se podría proponer añadir distintos decodificadores basados en otros algoritmos utilizando las clases aquí diseñadas, así como contribuir al entorno SageMath con estas.
