\chapter*{Conclusiones}%
\label{chap:conclusiones}

Como dijimos en la introducción de este trabajo, los objetivos de este eran estudiar los fundamentos de los códigos lineales, las extensiones de Ore, y la versión del algoritmo de Sugiyama para la familia de códigos skew RS, así como llevar a cabo la implementación del algoritmo y la representación de estos códigos en el entorno SageMath.

Podemos concluir que todos estos objetivos han sido alcanzados. En primer lugar presentamos los códigos lineales, explicando su estructura y como se aprovecha esta en el proceso de transmisión de un mensaje. Seguidamente estudiamos la extensiones de Ore, que son la base de la familia de códigos que utiliza la versión del algoritmo de Sugiyama que aquí presentamos. Tras este estudio, introducimos la familia de códigos cíclicos sesgados, y en particular la de códigos skew RS, y tras el análisis de estas pudimos mostrar el algoritmo de Sugiyama para códigos skew RS y demostrar su correcto funcionamiento. Por último, una vez comprendido el fundamento teórico de este algoritmo, este fue implementada en el entorno SageMath junto con las estructuras necesarias para representar dicha familia de códigos, y cuya documentación se encuentra incluida en el último capítulo.

Como futuro trabajo se podría proponer añadir distintos decodificadores basados en otros algoritmos utilizando las clases aquí diseñadas, así como contribuir al entorno SageMath con estas.
