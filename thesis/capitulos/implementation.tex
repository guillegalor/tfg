\chapter{Implementación en SageMath del Algoritmo de Sugiyama para códigos cíclicos sesgados}%
\label{chap:implementación_en_sagemath_del_algoritmo_de_sugiyama_para_códigos_cíclicos_sesgados}

El desarrollo del algoritmo de Sugiyama para códigos RS sesgados, juntos con la estructura necesaria para representar dichos códigos se ha realizado en el entorno SageMath (\cite{sagemath}). Este es una implementación de código abierto de software matemático y científico basado en Python.

Ya existen en SageMath clases esqueleto para representar códigos lineales, así como para implementar métodos de codificación y decodificación para estos. Por ello, para nuestra implementación hemos seguido las directrices de presentes en
\begin{center}
\url{https://doc.sagemath.org/html/en/thematic_tutorials/structures_in_coding_theory.html}.
\end{center}
Además de la estructura base para códigos, Sage también contiene una implementación de anillos de polinomios sesgados, la cual se usa a lo largo de todo el desarrollo del algoritmo y en las clases esqueleto para los nuevos códigos.

La primera clase que se ha implementado es \texttt{SkewCyclicCode} para representar cualquier código cíclico sesgado, la cual hereda de la clase abstracta \texttt{AbstractLinearCode}. Para poder codificar mensajes en palabras código de este tipo de códigos se han diseñado las clases \texttt{SkewCyclicCodeVectorEncoder} y \texttt{SkewCyclicCodePolynomialEncoder}, ambas heredando de la clase de Sage \texttt{Encoder}. A continuación, heredando de la clase \texttt{SkewCyclicCode} se implementó la clase \texttt{SkewRSCode} que almacena la estructura necesaria para trabajar con códigos Reed-Solomon sesgados. Por último, la clase \texttt{SkewRSCodeSugiyamaDecoder}, que hereda de la clase \texttt{Decoder}, contiene la implementación del algoritmo de decodificación desarrollado en este trabajo.

El código fuente se encuentra en el archivo \texttt{skew\_cyclic\_code.sage} en
\begin{center}
\url{https://github.com/guillegalor/tfg/tree/master/code}.
\end{center}
En la misma carpeta se encuentra otro archivo con ejemplos para probar las clases implementadas.

A continuación mostramos la documentación junto con ejemplos de uso de cada una de las estructuras y funciones implementadas.

\newpage
\section{Clase de códigos cíclicos sesgados}%
\label{sec:clase_de_códigos_cíclicos_sesgados}

\begin{description}
    \item[class SkewCyclicCode(self, generator\_pol=None)]
    Clase para representar un código cíclico sesgado. Hereda de \texttt{AbstractLinearCode}.

    \textsc{Argumentos}
    \begin{description}
        \item[generator\_pol] Polinomio generador para construir el código. Debe dividir a \(x^{n} -1\), donde \(n\) es el grado del automorfismo asociado al anillo de polonimios sesgado al que pertenece \texttt{g}.
    \end{description}

    \textsc{Ejemplos}
    \begin{lstlisting}
sage: F.<t> = GF(3^10)
sage: sigma = F.frobenius_endomorphism()
sage: R.<x> = F['x', sigma]
sage: g = x**2 + t**24561*x + t**47264
sage: C = SkewCyclicCode(generator_pol=g)
sage: C
[10, 8] Skew Cyclic Code over Finite Field in t of size 3^10
    \end{lstlisting}

    \begin{lstlisting}
sage: F.<t> = GF(3^10)
sage: sigma = F.frobenius_endomorphism()
sage: R.<x> = F['x', sigma]
sage: g = x**3 - 1
sage: C = SkewCyclicCode(generator_pol=g)
Traceback (most recent call last)
...
ValueError: Provided polynomial must divide x^n - 1, where n is the order of the ring automorphism.
    \end{lstlisting}

    \textsc{Métodos}
    \begin{description}
    \item[generator\_polynomial(self)]
    Devuelve el polinomio generador de \texttt{self}.

    \textsc{Ejemplos}
    \begin{lstlisting}
sage: F.<t> = GF(3^10)
sage: sigma = F.frobenius_endomorphism()
sage: R.<x> = F['x', sigma]
sage: g = x**2 + t**24561*x + t**47264
sage: C = SkewCyclicCode(generator_pol=g)
sage: C.generator_polynomial()
x^2 + ...  + t^5 + t^2 + 2
    \end{lstlisting}

    \end{description}

    \begin{description}
    \item[skew\_polynomial\_ring(self)]
    Devuelve el anillo de polinomios de sesgados de \texttt{self}.

    \textsc{Ejemplos}
    \begin{lstlisting}
sage: F.<t> = GF(3^10)
sage: sigma = F.frobenius_endomorphism()
sage: R.<x> = F['x', sigma]
sage: g = x**2 + t**24561*x + t**47264
sage: C = SkewCyclicCode(generator_pol=g)
sage: C.skew_polynomial_ring()
Skew Polynomial Ring in x over Finite Field in t of size 3^10 twisted by t |--> t^3
    \end{lstlisting}

    \end{description}

\end{description}

\section{Clases de codificadores para códigos cíclicos sesgados}

\begin{description}
    \item[class SkewCyclicCodeVectorEncoder(self, code)]
    Clase para codificar un vector como palabras código. Hereda de \texttt{Encoder}.

    \textsc{Argumentos}

    \begin{description}
        \item[code]
        Código sobre el que codificaremos los vectores.

    \end{description}

    \textsc{Ejemplos}

    \begin{lstlisting}
sage: F.<t> = GF(3^10)
sage: sigma = F.frobenius_endomorphism()
sage: R.<x> = F['x', sigma]
sage: g = x**2 + t**24561*x + t**47264
sage: C = SkewCyclicCode(generator_pol=g)
sage: E = SkewCyclicCodeVectorEncoder(C)
sage: E
Vector-style encoder for [10, 8] Skew Cyclic Code over Finite Field in t of size 3^10
    \end{lstlisting}

    \textsc{Métodos}

    \begin{description}
        \item[generator\_matrix(self)]
        Construye una matriz generatriz asociada al código asociado a este codificador

        \textsc{Ejemplos}
    \begin{lstlisting}
sage: F.<t> = GF(3^10)
sage: sigma = F.frobenius_endomorphism()
sage: R.<x> = F['x', sigma]
sage: g = x**2 + t**24561*x + t**47264
sage: C = SkewCyclicCode(generator_pol=g)
sage: E = SkewCyclicCodeVectorEncoder(C)
sage: E.generator_matrix()
[2*t^9 + t^8 + 2*t^6 + t^5 + t^2 + 2 ... 0]
[                   ...                   ]
[                   ...                   ]
[                   ...                   ]
[                   ...                   ]
[                   ...                   ]
[                   ...                   ]
[0                  ...                  1]
    \end{lstlisting}
    \end{description}

\end{description}

\begin{description}
    \item[class SkewCyclicCodePolynomialEncoder(self, code)]
    Clase para codificar polinomios en palabras código. Hereda de \texttt{Encoder}.

    \textsc{Argumentos}
    \begin{description}
        \item[code]
        Código sobre el que codificaremos los vectores.
    \end{description}

    \textsc{Ejemplos}
    \begin{lstlisting}
sage: F.<t> = GF(3^10)
sage: sigma = F.frobenius_endomorphism()
sage: R.<x> = F['x', sigma]
sage: g = x**2 + t**24561*x + t**47264
sage: C = SkewCyclicCode(generator_pol=g)
sage: E = SkewCyclicCodePolynomialEncoder(C)
sage: E
Polynomial-style encoder for [10, 8] Skew Cyclic Code over Finite Field in t of size 3^10
    \end{lstlisting}

    \textsc{Métodos}
    \begin{description}
        \item[encode(self, p)]
        Transforma el polinomio \texttt{p} en un elemento del código asociado a \texttt{self}.

        \textsc{Argumentos}
        \begin{description}
            \item[p]
            Polinomio que será codificado.
        \end{description}

        \textsc{Ejemplos}
        \begin{lstlisting}
sage: F.<t> = GF(3^10)
sage: sigma = F.frobenius_endomorphism()
sage: R.<x> = F['x', sigma]
sage: g = x**2 + t**24561*x + t**47264
sage: C = SkewCyclicCode(generator_pol=g)
sage: E = SkewCyclicCodePolynomialEncoder(C)
sage: m = x + t + 1
sage: E.encode(m)
(t^8 + 2*t^7 + 2*t^6 + 2*t^4 + t^3 + t^2 + 1, ... , 0)
        \end{lstlisting}

        \item[unencode\_nocheck(self, c)]
        Devuelve el mensaje en forma de polinomio asociado a \texttt{c} sin comprobar si \texttt{c} pertenece al código

        \textsc{Argumentos}
        \begin{description}
            \item[c]
            Un vector de la misma longitud del código asociado a \texttt{self}
        \end{description}

        \textsc{Ejemplos}
        \begin{lstlisting}
sage: F.<t> = GF(3^10)
sage: sigma = F.frobenius_endomorphism()
sage: R.<x> = F['x', sigma]
sage: g = x**2 + t**24561*x + t**47264
sage: C = SkewCyclicCode(generator_pol=g)
sage: E = SkewCyclicCodePolynomialEncoder(C)
sage: m = x + t + 1
sage: w = E.encode(m)
sage: E.unencode_nocheck(w)
x + t + 1
        \end{lstlisting}

        \item[message\_space(self)]
        Devuelve el espacio de mensajes asociado a \texttt{self}.

        \textsc{Ejemplos}
        \begin{lstlisting}
sage: F.<t> = GF(3^10)
sage: sigma = F.frobenius_endomorphism()
sage: R.<x> = F['x', sigma]
sage: g = x**2 + t**24561*x + t**47264
sage: C = SkewCyclicCode(generator_pol=g)
sage: E = SkewCyclicCodePolynomialEncoder(C)
sage: E.message_space()
Skew Polynomial Ring in x over Finite Field in t of size 3^10 twisted by t |--> t^3
        \end{lstlisting}
    \end{description}

\end{description}

\section{Clase de Códigos Reed-Solomon Sesgados}

\begin{description}
    \item[class SkewRSCode(self, hamming\_dist=None, \\skew\_polynomial\_ring=None, alpha=None)]

    Clase para representar un código RS sesgado. Hereda de \texttt{SkewCyclicCode}.

    \textsc{Argumentos}
    \begin{description}
        \item[hamming\_dist] La distancia hamming de \texttt{self}.
        \item[skew\_polynomial\_ring] El anillo de polinomios sesgados base.
        \item[alpha] Un generador normal de la extensión de cuerpos sobre el cuerpo fijo por el automorfismo asociado a \texttt{skew\_polynomial\_ring}.
    \end{description}

    \textsc{Ejemplos}
    \begin{lstlisting}
sage: F.<t> = GF(2^12, modulus=x**12 + x**7 +
              x**6 + x**5 + x**3 + x +1)
sage: sigma = (F.frobenius_endomorphism())**10
sage: R.<x> = F['x', sigma]
sage: alpha = t
sage: RS_C = SkewRSCode(hamming_dist=5, skew_polynomial_ring=R, alpha=alpha)
sage: RS_C
[6, 2] Skew Reed Solomon Code over Finite
Field in t of size 2^12
    \end{lstlisting}

    \begin{lstlisting}
sage: F.<t> = GF(3^10)
sage: sigma = F.frobenius_endomorphism()
sage: R.<x> = F['x', sigma]
sage: RS_C = SkewRSCode(hamming_dist=4, skew_polynomial_ring=R, alpha=t)
ValueError: Provided alpha must be an normal generator of the field extension given by sigma
\end{lstlisting}

    \textsc{Métodos}
    \begin{description}
        \item[hamming\_dist(self)]
            Devuelve la distancia Hamming asociada a este código.

        \textsc{Ejemplos}
        \begin{lstlisting}
sage: F.<t> = GF(2^12, modulus=x**12 + x**7 +
              x**6 + x**5 + x**3 + x +1)
sage: sigma = (F.frobenius_endomorphism())**10
sage: R.<x> = F['x', sigma]
sage: alpha = t
sage: RS_C = SkewRSCode(hamming_dist=5, skew_polynomial_ring=R, alpha=alpha)
sage: RS_C.hamming_dist()
2
        \end{lstlisting}
    \end{description}
\end{description}

\section{Clase del decodificador para códigos RS sesgados}
\begin{description}
    \item[class SkewRSCodeSugiyamaDecoder(self, code)]
    Clase para decodificar un vector recibido en una palabra del código asociado a \texttt{self}. Hereda de \texttt{Decoder}.

    \textsc{Argumentos}
    \begin{description}
        \item[code]
        Código al que pertenecerá la palabra que devuelve el decodificador.
    \end{description}

    \textsc{Ejemplos}
    \begin{lstlisting}
sage: F.<t> = GF(2^12, modulus=x**12 + x**7 +
              x**6 + x**5 + x**3 + x +1)
sage: sigma = (F.frobenius_endomorphism())**10
sage: R.<x> = F['x', sigma]
sage: alpha = t
sage: RS_C = SkewRSCode(hamming_dist=5, skew_polynomial_ring=R, alpha=alpha)
sage: D = SkewRSCodeSugiyamaDecoder(RS_C)
sage: D
Decoder through the Sugiyama like algorithm of the [6, 2] Skew Reed Solomon Code over Finite Field in t
of size 2^12
    \end{lstlisting}

    \textsc{Métodos}
    \begin{description}
        \item[decode\_to\_code(self, word)]
        Decodifica el vector \texttt{word} en un elemento del código asociado a \texttt{self}.

        \textsc{Argumentos}
        \begin{description}
            \item[word]
            Vector de elemtos del mismo cuerpo sobre el que está construido el anillo de polinomios
            sesgados del código asociado a \texttt{self}.
        \end{description}

        \textsc{Ejemplos}
        \begin{lstlisting}
sage: F.<t> = GF(2^12, modulus=x**12 + x**7 +
              x**6 + x**5 + x**3 + x +1)
sage: sigma = (F.frobenius_endomorphism())**10
sage: R.<x> = F['x', sigma]
sage: alpha = t
sage: RS_C = SkewRSCode(hamming_dist=5, skew_polynomial_ring=R, alpha=alpha)
sage: D = SkewRSCodeSugiyamaDecoder(RS_C)
sage: P_E = SkewCyclicCodePolynomialEncoder(RS_C)
sage: m = x + t
sage: codeword = P_E.encode(m)
sage: noisy_codeword = copy(codeword)
sage: noisy_codeword[3] = t**671
sage: decoded_word = D.decode_to_code(noisy_codeword)
sage: codeword == decoded_word
True
        \end{lstlisting}

        \item[decoding\_radius(self)]
        Devuelve el número de errores que \texttt{self} es capaz de corregir.

        \textsc{Ejemplos}
        \begin{lstlisting}
sage: F.<t> = GF(2^12, modulus=x**12 + x**7 +
              x**6 + x**5 + x**3 + x +1)
sage: sigma = (F.frobenius_endomorphism())**10
sage: R.<x> = F['x', sigma]
sage: alpha = t
sage: RS_C = SkewRSCode(hamming_dist=5, skew_polynomial_ring=R, alpha=alpha)
sage: D = SkewRSCodeSugiyamaDecoder(RS_C)
sage: D.decoding_radius()
2
        \end{lstlisting}
    \end{description}
\end{description}

\section{Funciones auxiliares}

\begin{description}
    \item[left\_extended\_euclidean\_algorithm(skew\_polynomial\_ring, f, g)]

    Implementación del algoritmo extendido de Euclides a la izquierda.

    \textsc{Argumentos}
    \begin{description}
        \item[skew\_polynomial\_ring] anillo de polinomios sesgados
        \item[f] polinomio del anillo pasado como argumento
        \item[g] polinomio del anillo pasado como argumento de grado menor que \texttt{f}
    \end{description}

    \textsc{Salida}
    \begin{description}
        \item[u, v r] vectores de polinomios sesgados tales que \texttt{u[i]*f + v[i]*g = r[i]} para cualquier posición \texttt{i}
    \end{description}

    \textsc{Ejemplos}
    \begin{lstlisting}
sage: F.<t> = GF(3^10)
sage: sigma = F.frobenius_endomorphism()
sage: R.<x> = F['x', sigma]
sage: a = x + t +1;
sage: b = t^2 * x**2 + (t+1)*x +1
sage: f = a*b
sage: g = b
sage: left_extended_euclidean_algorithm(R, f, g)
([1, 0, 1], [0, 1, 2*x + 2*t + 2], [t^6*x^3 + (2*t^3 + t^2 + 1)*x^2 + (t^2 + 2*t + 2)*x + t + 1, t^2*x^2 + (t + 1)*x + 1, 0])
    \end{lstlisting}

    \item[right\_extended\_euclidean\_algorithm(skew\_polynomial\_ring, f, g)]

    Implementación del algoritmo extendido de Euclides a la derecha.

    \textsc{Argumentos}
    \begin{description}
        \item[skew\_polynomial\_ring] anillo de polinomios sesgados
        \item[f] polinomio del anillo pasado como argumento
        \item[g] polinomio del anillo pasado como argumento de grado menor que \texttt{f}
    \end{description}

    \textsc{Salida}
    \begin{description}
        \item[u, v r] vectores de polinomios sesgados tales que \texttt{f*u[i] + g*v[i] = r[i]} para cualquier posición \texttt{i}
    \end{description}

    \textsc{Ejemplos}
    \begin{lstlisting}
sage: F.<t> = GF(3^10)
sage: sigma = F.frobenius_endomorphism()
sage: R.<x> = F['x', sigma]
sage: a = x + t +1;
sage: b = t^2 * x**2 + (t+1)*x +1
sage: f = a*b
sage: g = a
sage: right_extended_euclidean_algorithm(R, f, g)
([1, 0, 1], [0, 1, 2*t^2*x^2 + (2*t + 2)*x + 2], [t^6*x^3 + (2*t^3 + t^2 + 1)*x^2 + (t^2 + 2*t + 2)*x + t + 1, x + t + 1, 0])
    \end{lstlisting}

    \item[left\_lcm(pols)]
    Calcula el mínimo común múltiplo a la izquierda para una lista de polinomios sesgados

    \textsc{Argumentos}
    \begin{description}
        \item[pols] Lista de polinomios sesgados
    \end{description}

    \textsc{Ejemplos}
    \begin{lstlisting}
sage: F.<t> = GF(3^10)
sage: sigma = F.frobenius_endomorphism()
sage: R.<x> = F['x', sigma]
sage: a = x + t +1;
sage: b = t^2 * x**2 + (t+1)*x +1
sage: left_lcm((a,b))
x^3 + (2*t^9 + 2*t^8 + 2*t^7 + t^6 + t^5 + 2*t^4 + t^2 + t)*x^2 + (2*t^9 + t^8 + 2*t^7 + 2*t^6 + 2*t^5 + 2*t^3 + t)*x + t^8 + t^6 + 2*t^4 + t^3 + 2*t + 2
    \end{lstlisting}

    \item[norm(j, sigma, gamma)]
    Calcula la norma \texttt{j}-ésima de \texttt{gamma}.

    \textsc{Argumentos}
    \begin{description}
        \item[j] Número entero
        \item[sigma] Automorfismo del cuerpo donde esté definido \texttt{gamma}
        \item[gamma] Un elemento de un cuerpo cualquiera
    \end{description}

    \textsc{Ejemplos}
    \begin{lstlisting}
sage: F.<t> = GF(3^10)
sage: sigma = F.frobenius_endomorphism()
sage: R.<x> = F['x', sigma]
sage: norm(2, sigma, t**2 + t)
t^8 + t^7 + t^5 + t^4
    \end{lstlisting}

\end{description}
