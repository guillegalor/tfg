% Plantilla para un Trabajo Fin de Grado de la Universidad de Granada,
% adaptada para el Doble Grado en Ingeniería Informática y Matemáticas.
%
%  Autor: Mario Román.
%  Licencia: GNU GPLv2.
%
% Esta plantilla es una adaptación al castellano de la plantilla
% classicthesis de André Miede, que puede obtenerse en:
%  https://ctan.org/tex-archive/macros/latex/contrib/classicthesis?lang=en
% La plantilla original se licencia en GNU GPLv2.
%
% Esta plantilla usa símbolos de la Universidad de Granada sujetos a la normativa
% de identidad visual corporativa, que puede encontrarse en:
% http://secretariageneral.ugr.es/pages/ivc/normativa
%
% La compilación se realiza con las siguientes instrucciones:
%   pdflatex --shell-escape main.tex
%   bibtex main
%   pdflatex --shell-escape main.tex
%   pdflatex --shell-escape main.tex

% Opciones del tipo de documento
\documentclass[oneside,openright,titlepage,numbers=noenddot,openany,headinclude,footinclude=true,
cleardoublepage=empty,abstractoff,BCOR=5mm,paper=a4,fontsize=12pt,main=spanish]{scrreprt}

% Paquetes de latex que se cargan al inicio. Cubren la entrada de
% texto, gráficos, código fuente y símbolos.
\usepackage[utf8]{inputenc}
\usepackage[T1]{fontenc}
\usepackage[spanish,ruled,vlined]{algorithm2e}
\usepackage{fixltx2e}
\usepackage{graphicx} % Inclusión de imágenes.
\usepackage{grffile}  % Distintos formatos para imágenes.
\usepackage{longtable} % Tablas multipágina.
\usepackage{wrapfig} % Coloca texto alrededor de una figura.
\usepackage{rotating}
\usepackage[normalem]{ulem}
\usepackage{amsmath}
\usepackage{textcomp}
\usepackage{amssymb}
\usepackage{capt-of}
\usepackage[colorlinks=true]{hyperref}
\usepackage{tikz} % Diagramas conmutativos.
\usepackage{minted} % Código fuente.
\usepackage[T1]{fontenc}
\usepackage{natbib}
\usepackage{enumerate}

% Plantilla classicthesis
\usepackage[beramono,eulerchapternumbers,linedheaders,parts,a5paper,dottedtoc,
manychapters,pdfspacing]{classicthesis}

% Geometría y espaciado de párrafos.
\setcounter{secnumdepth}{0}
\usepackage{enumitem}
\setitemize{noitemsep,topsep=0pt,parsep=0pt,partopsep=0pt}
\setlist[enumerate]{topsep=0pt,itemsep=-1ex,partopsep=1ex,parsep=1ex}
\usepackage[top=1in, bottom=1.5in, left=1in, right=1in]{geometry}
\setlength\itemsep{0em}
\setlength{\parindent}{0pt}
\usepackage{parskip}

% Profundidad de la tabla de contenidos.
\setcounter{secnumdepth}{3}

% Usa el paquete minted para mostrar trozos de código.
% Pueden seleccionarse el lenguaje apropiado y el estilo del código.
\usepackage{minted}
\usemintedstyle{colorful}
\setminted{fontsize=\small}
\setminted[haskell]{linenos=false,fontsize=\small}
\renewcommand{\theFancyVerbLine}{\sffamily\textcolor[rgb]{0.5,0.5,1.0}{\oldstylenums{\arabic{FancyVerbLine}}}}

% Archivos de configuración.
%------------------------
% Bibliotecas para matemáticas de latex
%------------------------
\usepackage{amsthm}
\usepackage{amsmath}
\usepackage{tikz}
\usepackage{tikz-cd}
\usetikzlibrary{shapes,fit}
\usepackage{bussproofs}
\EnableBpAbbreviations{}
\usepackage{mathtools}
\usepackage{scalerel}
\usepackage{stmaryrd}

%------------------------
% Estilos para los teoremas
%------------------------
\theoremstyle{plain}
\newtheorem{theorem}{Teorema}
\newtheorem{proposition}[theorem]{Proposición}
\newtheorem{lemma}[theorem]{Lema}
\newtheorem{corollary}[theorem]{Corolario}
\theoremstyle{definition}
\newtheorem{definition}[theorem]{Definición}
\newtheorem{proofs}{Demostración}
\theoremstyle{remark}
\newtheorem{remark}[theorem]{Comentario}
\newtheorem{exampleth}[theorem]{Ejemplo}

\begingroup\makeatletter\@for\theoremstyle:=definition,remark,plain\do{\expandafter\g@addto@macro\csname th@\theoremstyle\endcsname{\addtolength\thm@preskip\parskip}}\endgroup

%------------------------
% Macros
% ------------------------

% Aquí pueden añadirse abreviaturas para comandos de latex
% frequentemente usados.
\newcommand*\diff{\mathop{}\!\mathrm{d}}

% Comandos tomados de la plantilla de libreim

% Redefinir letra griega épsilon.
\let\epsilon\upvarepsilon

% Valor absoluto: \abs{}
\providecommand{\abs}[1]{\lvert#1\rvert}

% Fracción grande: \ddfrac{}{}
\newcommand\ddfrac[2]{\frac{\displaystyle #1}{\displaystyle #2}}

% Texto en negrita en modo matemática: \bm{}
\newcommand{\bm}[1]{\boldsymbol{#1}}

% Letras de conjuntos
\newcommand{\R}{\mathbb{R}}
\newcommand{\N}{\mathbb{N}}
\newcommand{\F}{\mathbb{F}}
\newcommand{\C}{\mathcal{C}}

% Sucesiones
\newcommand{\xn}{\{x_n\}}
\newcommand{\fn}{\{f_n\}}

% Letra griega "chi" en línea con el texto
\DeclareRobustCommand{\rchi}{{\Large \mathpalette\irchi\relax}}
\newcommand{\irchi}[2]{\raisebox{0.4\depth}{$#1\chi$}} % inner command, used by \rchi

% Letra 'omega'
\newcommand{\W}{\Omega}
\newcommand{\w}{\omega}

% Conjunto vacío
\renewcommand{\O}{\emptyset}

% Coeficiente Lider
\newcommand{\lc}{\operatorname{lc}}

% Probabilidad
\newcommand{\prob}{\operatorname{prob}}

% Listas ordenadas con números romanos (i), (ii), etc.
\newenvironment{nlist}
{\begin{enumerate}
    \renewcommand\labelenumi{(\emph{\roman{enumi})}}}
  {\end{enumerate}}
  % En macros.tex se almacenan las opciones y comandos para escribir matemáticas.
% ****************************************************************************************************
% classicthesis-config.tex 
% formerly known as loadpackages.sty, classicthesis-ldpkg.sty, and classicthesis-preamble.sty 
% Use it at the beginning of your ClassicThesis.tex, or as a LaTeX Preamble 
% in your ClassicThesis.{tex,lyx} with % ****************************************************************************************************
% classicthesis-config.tex 
% formerly known as loadpackages.sty, classicthesis-ldpkg.sty, and classicthesis-preamble.sty 
% Use it at the beginning of your ClassicThesis.tex, or as a LaTeX Preamble 
% in your ClassicThesis.{tex,lyx} with % ****************************************************************************************************
% classicthesis-config.tex 
% formerly known as loadpackages.sty, classicthesis-ldpkg.sty, and classicthesis-preamble.sty 
% Use it at the beginning of your ClassicThesis.tex, or as a LaTeX Preamble 
% in your ClassicThesis.{tex,lyx} with \input{classicthesis-config}
% ****************************************************************************************************  
% If you like the classicthesis, then I would appreciate a postcard. 
% My address can be found in the file ClassicThesis.pdf. A collection 
% of the postcards I received so far is available online at 
% http://postcards.miede.de
% ****************************************************************************************************


% ****************************************************************************************************
% 0. Set the encoding of your files. UTF-8 is the only sensible encoding nowadays. If you can't read
% äöüßáéçèê∂åëæƒÏ€ then change the encoding setting in your editor, not the line below. If your editor
% does not support utf8 use another editor!
% ****************************************************************************************************
\PassOptionsToPackage{utf8x}{inputenc}
	\usepackage{inputenc}

% ****************************************************************************************************
% 1. Configure classicthesis for your needs here, e.g., remove "drafting" below 
% in order to deactivate the time-stamp on the pages
% ****************************************************************************************************
\PassOptionsToPackage{eulerchapternumbers,listings,drafting,%
		pdfspacing,%floatperchapter,%linedheaders,%
                subfig,beramono,eulermath,parts,dottedtoc}{classicthesis}                                        
% ********************************************************************
% Available options for classicthesis.sty 
% (see ClassicThesis.pdf for more information):
% drafting
% parts nochapters linedheaders
% eulerchapternumbers beramono eulermath pdfspacing minionprospacing
% tocaligned dottedtoc manychapters
% listings floatperchapter subfig
% ********************************************************************

% ****************************************************************************************************
% 2. Personal data and user ad-hoc commands
% ****************************************************************************************************
\newcommand{\myTitle}{A Classic Thesis Style\xspace}
\newcommand{\mySubtitle}{An Homage to The Elements of Typographic Style\xspace}
\newcommand{\myDegree}{Doktor-Ingenieur (Dr.-Ing.)\xspace}
\newcommand{\myName}{André Miede\xspace}
\newcommand{\myProf}{Put name here\xspace}
\newcommand{\myOtherProf}{Put name here\xspace}
\newcommand{\mySupervisor}{Put name here\xspace}
\newcommand{\myFaculty}{Put data here\xspace}
\newcommand{\myDepartment}{Put data here\xspace}
\newcommand{\myUni}{Put data here\xspace}
\newcommand{\myLocation}{Saarbrücken\xspace}
\newcommand{\myTime}{September 2015\xspace}
%\newcommand{\myVersion}{version 4.2\xspace}

% ********************************************************************
% Setup, finetuning, and useful commands
% ********************************************************************
\newcounter{dummy} % necessary for correct hyperlinks (to index, bib, etc.)
\newlength{\abcd} % for ab..z string length calculation
\providecommand{\mLyX}{L\kern-.1667em\lower.25em\hbox{Y}\kern-.125emX\@}
\newcommand{\ie}{i.\,e.}
\newcommand{\Ie}{I.\,e.}
\newcommand{\eg}{e.\,g.}
\newcommand{\Eg}{E.\,g.} 
% ****************************************************************************************************


% ****************************************************************************************************
% 3. Loading some handy packages
% ****************************************************************************************************
% ******************************************************************** 
% Packages with options that might require adjustments
% ******************************************************************** 
%\PassOptionsToPackage{ngerman,american}{babel}   % change this to your language(s)
% Spanish languages need extra options in order to work with this template
% \PassOptionsToPackage{es-lcroman,spanish}{babel}
\usepackage[main=spanish]{babel}

%\usepackage{csquotes}
% \PassOptionsToPackage{%
%     %backend=biber, %instead of bibtex
% 	backend=bibtex8,bibencoding=ascii,%
% 	language=auto,%
% 	style=alpha,%
%     %style=authoryear-comp, % Author 1999, 2010
%     %bibstyle=authoryear,dashed=false, % dashed: substitute rep. author with ---
%     sorting=nyt, % name, year, title
%     maxbibnames=10, % default: 3, et al.
%     %backref=true,%
%     natbib=true % natbib compatibility mode (\citep and \citet still work)
% }{biblatex}
%     \usepackage{biblatex}

% \PassOptionsToPackage{fleqn}{amsmath}       % math environments and more by the AMS 
%     \usepackage{amsmath}

% ******************************************************************** 
% General useful packages
% ******************************************************************** 
\PassOptionsToPackage{T1}{fontenc} % T2A for cyrillics
    \usepackage{fontenc}     
\usepackage{textcomp} % fix warning with missing font shapes
\usepackage{scrhack} % fix warnings when using KOMA with listings package          
\usepackage{xspace} % to get the spacing after macros right  
\usepackage{mparhack} % get marginpar right
\usepackage{fixltx2e} % fixes some LaTeX stuff --> since 2015 in the LaTeX kernel (see below)
%\usepackage[latest]{latexrelease} % will be used once available in more distributions (ISSUE #107)
\PassOptionsToPackage{printonlyused,smaller}{acronym} 
    \usepackage{acronym} % nice macros for handling all acronyms in the thesis
    %\renewcommand{\bflabel}[1]{{#1}\hfill} % fix the list of acronyms --> no longer working
    %\renewcommand*{\acsfont}[1]{\textsc{#1}} 
    \renewcommand*{\aclabelfont}[1]{\acsfont{#1}}
% ****************************************************************************************************


% ****************************************************************************************************
% 4. Setup floats: tables, (sub)figures, and captions
% ****************************************************************************************************
\usepackage{tabularx} % better tables
    \setlength{\extrarowheight}{3pt} % increase table row height
\newcommand{\tableheadline}[1]{\multicolumn{1}{c}{\spacedlowsmallcaps{#1}}}
\newcommand{\myfloatalign}{\centering} % to be used with each float for alignment
\usepackage{caption}
% Thanks to cgnieder and Claus Lahiri
% http://tex.stackexchange.com/questions/69349/spacedlowsmallcaps-in-caption-label
% [REMOVED DUE TO OTHER PROBLEMS, SEE ISSUE #82]    
%\DeclareCaptionLabelFormat{smallcaps}{\bothIfFirst{#1}{~}\MakeTextLowercase{\textsc{#2}}}
%\captionsetup{font=small,labelformat=smallcaps} % format=hang,
\captionsetup{font=small} % format=hang,
\usepackage{subfig}  
% ****************************************************************************************************


% ****************************************************************************************************
% 5. Setup code listings
% ****************************************************************************************************
% \usepackage{listings} 
% %\lstset{emph={trueIndex,root},emphstyle=\color{BlueViolet}}%\underbar} % for special keywords
% \lstset{language={Haskell},morekeywords={PassOptionsToPackage,selectlanguage},keywordstyle=\color{RoyalBlue},basicstyle=\small\ttfamily,commentstyle=\color{Green}\ttfamily,stringstyle=\rmfamily,numbers=none,numberstyle=\scriptsize,stepnumber=5,numbersep=8pt,showstringspaces=false,breaklines=true,belowcaptionskip=.75\baselineskip} 
% ****************************************************************************************************             


% ****************************************************************************************************
% 6. PDFLaTeX, hyperreferences and citation backreferences
% ****************************************************************************************************
% ********************************************************************
% Using PDFLaTeX
% ********************************************************************
\PassOptionsToPackage{pdftex,hyperfootnotes=false,pdfpagelabels}{hyperref}
    \usepackage{hyperref}  % backref linktocpage pagebackref
\pdfcompresslevel=9
\pdfadjustspacing=1 
\PassOptionsToPackage{pdftex}{graphicx}
    \usepackage{graphicx} 
 

% ********************************************************************
% Hyperreferences
% ********************************************************************
\hypersetup{%
    %draft, % = no hyperlinking at all (useful in b/w printouts)
    colorlinks=true, linktocpage=true, pdfstartpage=3, pdfstartview=FitV,%
    % uncomment the following line if you want to have black links (e.g., for printing)
    %colorlinks=false, linktocpage=false, pdfstartpage=3, pdfstartview=FitV, pdfborder={0 0 0},%
    breaklinks=true, pdfpagemode=UseNone, pageanchor=true, pdfpagemode=UseOutlines,%
    plainpages=false, bookmarksnumbered, bookmarksopen=true, bookmarksopenlevel=1,%
    hypertexnames=true, pdfhighlight=/O,%nesting=true,%frenchlinks,%
    urlcolor=webbrown, linkcolor=RoyalBlue, citecolor=webgreen, %pagecolor=RoyalBlue,%
    %urlcolor=Black, linkcolor=Black, citecolor=Black, %pagecolor=Black,%
    pdftitle={\myTitle},%
    pdfauthor={\textcopyright\ \myName, \myUni, \myFaculty},%
    pdfsubject={},%
    pdfkeywords={},%
    pdfcreator={pdfLaTeX},%
    pdfproducer={LaTeX with hyperref and classicthesis}%
}   

% ********************************************************************
% Setup autoreferences
% ********************************************************************
% There are some issues regarding autorefnames
% http://www.ureader.de/msg/136221647.aspx
% http://www.tex.ac.uk/cgi-bin/texfaq2html?label=latexwords
% you have to redefine the makros for the 
% language you use, e.g., american, ngerman
% (as chosen when loading babel/AtBeginDocument)
% ********************************************************************
\makeatletter
\@ifpackageloaded{babel}%
    {%
       \addto\extrasamerican{%
			\renewcommand*{\figureautorefname}{Figure}%
			\renewcommand*{\tableautorefname}{Table}%
			\renewcommand*{\partautorefname}{Part}%
			\renewcommand*{\chapterautorefname}{Chapter}%
			\renewcommand*{\sectionautorefname}{Section}%
			\renewcommand*{\subsectionautorefname}{Section}%
			\renewcommand*{\subsubsectionautorefname}{Section}%     
                }%
       \addto\extrasngerman{% 
			\renewcommand*{\paragraphautorefname}{Absatz}%
			\renewcommand*{\subparagraphautorefname}{Unterabsatz}%
			\renewcommand*{\footnoteautorefname}{Fu\"snote}%
			\renewcommand*{\FancyVerbLineautorefname}{Zeile}%
			\renewcommand*{\theoremautorefname}{Theorem}%
			\renewcommand*{\appendixautorefname}{Anhang}%
			\renewcommand*{\equationautorefname}{Gleichung}%        
			\renewcommand*{\itemautorefname}{Punkt}%
                }%  
            % Fix to getting autorefs for subfigures right (thanks to Belinda Vogt for changing the definition)
            \providecommand{\subfigureautorefname}{\figureautorefname}%             
    }{\relax}
\makeatother


% ****************************************************************************************************
% 7. Last calls before the bar closes
% ****************************************************************************************************
% ********************************************************************
% Development Stuff
% ********************************************************************
\listfiles
%\PassOptionsToPackage{l2tabu,orthodox,abort}{nag}
%   \usepackage{nag}
%\PassOptionsToPackage{warning, all}{onlyamsmath}
%   \usepackage{onlyamsmath}

% ********************************************************************
% Last, but not least...
% ********************************************************************
\usepackage{classicthesis} 
% ****************************************************************************************************


% ****************************************************************************************************
% 8. Further adjustments (experimental)
% ****************************************************************************************************
% ********************************************************************
% Changing the text area
% ********************************************************************
%\linespread{1.05} % a bit more for Palatino
%\areaset[current]{312pt}{761pt} % 686 (factor 2.2) + 33 head + 42 head \the\footskip
%\setlength{\marginparwidth}{7em}%
%\setlength{\marginparsep}{2em}%

% ********************************************************************
% Using different fonts
% ********************************************************************
%\usepackage[oldstylenums]{kpfonts} % oldstyle notextcomp
%\usepackage[osf]{libertine}
%\usepackage[light,condensed,math]{iwona}
%\renewcommand{\sfdefault}{iwona}
%\usepackage{lmodern} % <-- no osf support :-(
%\usepackage{cfr-lm} % 
%\usepackage[urw-garamond]{mathdesign} <-- no osf support :-(
%\usepackage[default,osfigures]{opensans} % scale=0.95 
%\usepackage[sfdefault]{FiraSans}
% ****************************************************************************************************

% ****************************************************************************************************  
% If you like the classicthesis, then I would appreciate a postcard. 
% My address can be found in the file ClassicThesis.pdf. A collection 
% of the postcards I received so far is available online at 
% http://postcards.miede.de
% ****************************************************************************************************


% ****************************************************************************************************
% 0. Set the encoding of your files. UTF-8 is the only sensible encoding nowadays. If you can't read
% äöüßáéçèê∂åëæƒÏ€ then change the encoding setting in your editor, not the line below. If your editor
% does not support utf8 use another editor!
% ****************************************************************************************************
\PassOptionsToPackage{utf8x}{inputenc}
	\usepackage{inputenc}

% ****************************************************************************************************
% 1. Configure classicthesis for your needs here, e.g., remove "drafting" below 
% in order to deactivate the time-stamp on the pages
% ****************************************************************************************************
\PassOptionsToPackage{eulerchapternumbers,listings,drafting,%
		pdfspacing,%floatperchapter,%linedheaders,%
                subfig,beramono,eulermath,parts,dottedtoc}{classicthesis}                                        
% ********************************************************************
% Available options for classicthesis.sty 
% (see ClassicThesis.pdf for more information):
% drafting
% parts nochapters linedheaders
% eulerchapternumbers beramono eulermath pdfspacing minionprospacing
% tocaligned dottedtoc manychapters
% listings floatperchapter subfig
% ********************************************************************

% ****************************************************************************************************
% 2. Personal data and user ad-hoc commands
% ****************************************************************************************************
\newcommand{\myTitle}{A Classic Thesis Style\xspace}
\newcommand{\mySubtitle}{An Homage to The Elements of Typographic Style\xspace}
\newcommand{\myDegree}{Doktor-Ingenieur (Dr.-Ing.)\xspace}
\newcommand{\myName}{André Miede\xspace}
\newcommand{\myProf}{Put name here\xspace}
\newcommand{\myOtherProf}{Put name here\xspace}
\newcommand{\mySupervisor}{Put name here\xspace}
\newcommand{\myFaculty}{Put data here\xspace}
\newcommand{\myDepartment}{Put data here\xspace}
\newcommand{\myUni}{Put data here\xspace}
\newcommand{\myLocation}{Saarbrücken\xspace}
\newcommand{\myTime}{September 2015\xspace}
%\newcommand{\myVersion}{version 4.2\xspace}

% ********************************************************************
% Setup, finetuning, and useful commands
% ********************************************************************
\newcounter{dummy} % necessary for correct hyperlinks (to index, bib, etc.)
\newlength{\abcd} % for ab..z string length calculation
\providecommand{\mLyX}{L\kern-.1667em\lower.25em\hbox{Y}\kern-.125emX\@}
\newcommand{\ie}{i.\,e.}
\newcommand{\Ie}{I.\,e.}
\newcommand{\eg}{e.\,g.}
\newcommand{\Eg}{E.\,g.} 
% ****************************************************************************************************


% ****************************************************************************************************
% 3. Loading some handy packages
% ****************************************************************************************************
% ******************************************************************** 
% Packages with options that might require adjustments
% ******************************************************************** 
%\PassOptionsToPackage{ngerman,american}{babel}   % change this to your language(s)
% Spanish languages need extra options in order to work with this template
% \PassOptionsToPackage{es-lcroman,spanish}{babel}
\usepackage[main=spanish]{babel}

%\usepackage{csquotes}
% \PassOptionsToPackage{%
%     %backend=biber, %instead of bibtex
% 	backend=bibtex8,bibencoding=ascii,%
% 	language=auto,%
% 	style=alpha,%
%     %style=authoryear-comp, % Author 1999, 2010
%     %bibstyle=authoryear,dashed=false, % dashed: substitute rep. author with ---
%     sorting=nyt, % name, year, title
%     maxbibnames=10, % default: 3, et al.
%     %backref=true,%
%     natbib=true % natbib compatibility mode (\citep and \citet still work)
% }{biblatex}
%     \usepackage{biblatex}

% \PassOptionsToPackage{fleqn}{amsmath}       % math environments and more by the AMS 
%     \usepackage{amsmath}

% ******************************************************************** 
% General useful packages
% ******************************************************************** 
\PassOptionsToPackage{T1}{fontenc} % T2A for cyrillics
    \usepackage{fontenc}     
\usepackage{textcomp} % fix warning with missing font shapes
\usepackage{scrhack} % fix warnings when using KOMA with listings package          
\usepackage{xspace} % to get the spacing after macros right  
\usepackage{mparhack} % get marginpar right
\usepackage{fixltx2e} % fixes some LaTeX stuff --> since 2015 in the LaTeX kernel (see below)
%\usepackage[latest]{latexrelease} % will be used once available in more distributions (ISSUE #107)
\PassOptionsToPackage{printonlyused,smaller}{acronym} 
    \usepackage{acronym} % nice macros for handling all acronyms in the thesis
    %\renewcommand{\bflabel}[1]{{#1}\hfill} % fix the list of acronyms --> no longer working
    %\renewcommand*{\acsfont}[1]{\textsc{#1}} 
    \renewcommand*{\aclabelfont}[1]{\acsfont{#1}}
% ****************************************************************************************************


% ****************************************************************************************************
% 4. Setup floats: tables, (sub)figures, and captions
% ****************************************************************************************************
\usepackage{tabularx} % better tables
    \setlength{\extrarowheight}{3pt} % increase table row height
\newcommand{\tableheadline}[1]{\multicolumn{1}{c}{\spacedlowsmallcaps{#1}}}
\newcommand{\myfloatalign}{\centering} % to be used with each float for alignment
\usepackage{caption}
% Thanks to cgnieder and Claus Lahiri
% http://tex.stackexchange.com/questions/69349/spacedlowsmallcaps-in-caption-label
% [REMOVED DUE TO OTHER PROBLEMS, SEE ISSUE #82]    
%\DeclareCaptionLabelFormat{smallcaps}{\bothIfFirst{#1}{~}\MakeTextLowercase{\textsc{#2}}}
%\captionsetup{font=small,labelformat=smallcaps} % format=hang,
\captionsetup{font=small} % format=hang,
\usepackage{subfig}  
% ****************************************************************************************************


% ****************************************************************************************************
% 5. Setup code listings
% ****************************************************************************************************
% \usepackage{listings} 
% %\lstset{emph={trueIndex,root},emphstyle=\color{BlueViolet}}%\underbar} % for special keywords
% \lstset{language={Haskell},morekeywords={PassOptionsToPackage,selectlanguage},keywordstyle=\color{RoyalBlue},basicstyle=\small\ttfamily,commentstyle=\color{Green}\ttfamily,stringstyle=\rmfamily,numbers=none,numberstyle=\scriptsize,stepnumber=5,numbersep=8pt,showstringspaces=false,breaklines=true,belowcaptionskip=.75\baselineskip} 
% ****************************************************************************************************             


% ****************************************************************************************************
% 6. PDFLaTeX, hyperreferences and citation backreferences
% ****************************************************************************************************
% ********************************************************************
% Using PDFLaTeX
% ********************************************************************
\PassOptionsToPackage{pdftex,hyperfootnotes=false,pdfpagelabels}{hyperref}
    \usepackage{hyperref}  % backref linktocpage pagebackref
\pdfcompresslevel=9
\pdfadjustspacing=1 
\PassOptionsToPackage{pdftex}{graphicx}
    \usepackage{graphicx} 
 

% ********************************************************************
% Hyperreferences
% ********************************************************************
\hypersetup{%
    %draft, % = no hyperlinking at all (useful in b/w printouts)
    colorlinks=true, linktocpage=true, pdfstartpage=3, pdfstartview=FitV,%
    % uncomment the following line if you want to have black links (e.g., for printing)
    %colorlinks=false, linktocpage=false, pdfstartpage=3, pdfstartview=FitV, pdfborder={0 0 0},%
    breaklinks=true, pdfpagemode=UseNone, pageanchor=true, pdfpagemode=UseOutlines,%
    plainpages=false, bookmarksnumbered, bookmarksopen=true, bookmarksopenlevel=1,%
    hypertexnames=true, pdfhighlight=/O,%nesting=true,%frenchlinks,%
    urlcolor=webbrown, linkcolor=RoyalBlue, citecolor=webgreen, %pagecolor=RoyalBlue,%
    %urlcolor=Black, linkcolor=Black, citecolor=Black, %pagecolor=Black,%
    pdftitle={\myTitle},%
    pdfauthor={\textcopyright\ \myName, \myUni, \myFaculty},%
    pdfsubject={},%
    pdfkeywords={},%
    pdfcreator={pdfLaTeX},%
    pdfproducer={LaTeX with hyperref and classicthesis}%
}   

% ********************************************************************
% Setup autoreferences
% ********************************************************************
% There are some issues regarding autorefnames
% http://www.ureader.de/msg/136221647.aspx
% http://www.tex.ac.uk/cgi-bin/texfaq2html?label=latexwords
% you have to redefine the makros for the 
% language you use, e.g., american, ngerman
% (as chosen when loading babel/AtBeginDocument)
% ********************************************************************
\makeatletter
\@ifpackageloaded{babel}%
    {%
       \addto\extrasamerican{%
			\renewcommand*{\figureautorefname}{Figure}%
			\renewcommand*{\tableautorefname}{Table}%
			\renewcommand*{\partautorefname}{Part}%
			\renewcommand*{\chapterautorefname}{Chapter}%
			\renewcommand*{\sectionautorefname}{Section}%
			\renewcommand*{\subsectionautorefname}{Section}%
			\renewcommand*{\subsubsectionautorefname}{Section}%     
                }%
       \addto\extrasngerman{% 
			\renewcommand*{\paragraphautorefname}{Absatz}%
			\renewcommand*{\subparagraphautorefname}{Unterabsatz}%
			\renewcommand*{\footnoteautorefname}{Fu\"snote}%
			\renewcommand*{\FancyVerbLineautorefname}{Zeile}%
			\renewcommand*{\theoremautorefname}{Theorem}%
			\renewcommand*{\appendixautorefname}{Anhang}%
			\renewcommand*{\equationautorefname}{Gleichung}%        
			\renewcommand*{\itemautorefname}{Punkt}%
                }%  
            % Fix to getting autorefs for subfigures right (thanks to Belinda Vogt for changing the definition)
            \providecommand{\subfigureautorefname}{\figureautorefname}%             
    }{\relax}
\makeatother


% ****************************************************************************************************
% 7. Last calls before the bar closes
% ****************************************************************************************************
% ********************************************************************
% Development Stuff
% ********************************************************************
\listfiles
%\PassOptionsToPackage{l2tabu,orthodox,abort}{nag}
%   \usepackage{nag}
%\PassOptionsToPackage{warning, all}{onlyamsmath}
%   \usepackage{onlyamsmath}

% ********************************************************************
% Last, but not least...
% ********************************************************************
\usepackage{classicthesis} 
% ****************************************************************************************************


% ****************************************************************************************************
% 8. Further adjustments (experimental)
% ****************************************************************************************************
% ********************************************************************
% Changing the text area
% ********************************************************************
%\linespread{1.05} % a bit more for Palatino
%\areaset[current]{312pt}{761pt} % 686 (factor 2.2) + 33 head + 42 head \the\footskip
%\setlength{\marginparwidth}{7em}%
%\setlength{\marginparsep}{2em}%

% ********************************************************************
% Using different fonts
% ********************************************************************
%\usepackage[oldstylenums]{kpfonts} % oldstyle notextcomp
%\usepackage[osf]{libertine}
%\usepackage[light,condensed,math]{iwona}
%\renewcommand{\sfdefault}{iwona}
%\usepackage{lmodern} % <-- no osf support :-(
%\usepackage{cfr-lm} % 
%\usepackage[urw-garamond]{mathdesign} <-- no osf support :-(
%\usepackage[default,osfigures]{opensans} % scale=0.95 
%\usepackage[sfdefault]{FiraSans}
% ****************************************************************************************************

% ****************************************************************************************************  
% If you like the classicthesis, then I would appreciate a postcard. 
% My address can be found in the file ClassicThesis.pdf. A collection 
% of the postcards I received so far is available online at 
% http://postcards.miede.de
% ****************************************************************************************************


% ****************************************************************************************************
% 0. Set the encoding of your files. UTF-8 is the only sensible encoding nowadays. If you can't read
% äöüßáéçèê∂åëæƒÏ€ then change the encoding setting in your editor, not the line below. If your editor
% does not support utf8 use another editor!
% ****************************************************************************************************
\PassOptionsToPackage{utf8x}{inputenc}
	\usepackage{inputenc}

% ****************************************************************************************************
% 1. Configure classicthesis for your needs here, e.g., remove "drafting" below 
% in order to deactivate the time-stamp on the pages
% ****************************************************************************************************
\PassOptionsToPackage{eulerchapternumbers,listings,drafting,%
		pdfspacing,%floatperchapter,%linedheaders,%
                subfig,beramono,eulermath,parts,dottedtoc}{classicthesis}                                        
% ********************************************************************
% Available options for classicthesis.sty 
% (see ClassicThesis.pdf for more information):
% drafting
% parts nochapters linedheaders
% eulerchapternumbers beramono eulermath pdfspacing minionprospacing
% tocaligned dottedtoc manychapters
% listings floatperchapter subfig
% ********************************************************************

% ****************************************************************************************************
% 2. Personal data and user ad-hoc commands
% ****************************************************************************************************
\newcommand{\myTitle}{A Classic Thesis Style\xspace}
\newcommand{\mySubtitle}{An Homage to The Elements of Typographic Style\xspace}
\newcommand{\myDegree}{Doktor-Ingenieur (Dr.-Ing.)\xspace}
\newcommand{\myName}{André Miede\xspace}
\newcommand{\myProf}{Put name here\xspace}
\newcommand{\myOtherProf}{Put name here\xspace}
\newcommand{\mySupervisor}{Put name here\xspace}
\newcommand{\myFaculty}{Put data here\xspace}
\newcommand{\myDepartment}{Put data here\xspace}
\newcommand{\myUni}{Put data here\xspace}
\newcommand{\myLocation}{Saarbrücken\xspace}
\newcommand{\myTime}{September 2015\xspace}
%\newcommand{\myVersion}{version 4.2\xspace}

% ********************************************************************
% Setup, finetuning, and useful commands
% ********************************************************************
\newcounter{dummy} % necessary for correct hyperlinks (to index, bib, etc.)
\newlength{\abcd} % for ab..z string length calculation
\providecommand{\mLyX}{L\kern-.1667em\lower.25em\hbox{Y}\kern-.125emX\@}
\newcommand{\ie}{i.\,e.}
\newcommand{\Ie}{I.\,e.}
\newcommand{\eg}{e.\,g.}
\newcommand{\Eg}{E.\,g.} 
% ****************************************************************************************************


% ****************************************************************************************************
% 3. Loading some handy packages
% ****************************************************************************************************
% ******************************************************************** 
% Packages with options that might require adjustments
% ******************************************************************** 
%\PassOptionsToPackage{ngerman,american}{babel}   % change this to your language(s)
% Spanish languages need extra options in order to work with this template
% \PassOptionsToPackage{es-lcroman,spanish}{babel}
\usepackage[main=spanish]{babel}

%\usepackage{csquotes}
% \PassOptionsToPackage{%
%     %backend=biber, %instead of bibtex
% 	backend=bibtex8,bibencoding=ascii,%
% 	language=auto,%
% 	style=alpha,%
%     %style=authoryear-comp, % Author 1999, 2010
%     %bibstyle=authoryear,dashed=false, % dashed: substitute rep. author with ---
%     sorting=nyt, % name, year, title
%     maxbibnames=10, % default: 3, et al.
%     %backref=true,%
%     natbib=true % natbib compatibility mode (\citep and \citet still work)
% }{biblatex}
%     \usepackage{biblatex}

% \PassOptionsToPackage{fleqn}{amsmath}       % math environments and more by the AMS 
%     \usepackage{amsmath}

% ******************************************************************** 
% General useful packages
% ******************************************************************** 
\PassOptionsToPackage{T1}{fontenc} % T2A for cyrillics
    \usepackage{fontenc}     
\usepackage{textcomp} % fix warning with missing font shapes
\usepackage{scrhack} % fix warnings when using KOMA with listings package          
\usepackage{xspace} % to get the spacing after macros right  
\usepackage{mparhack} % get marginpar right
\usepackage{fixltx2e} % fixes some LaTeX stuff --> since 2015 in the LaTeX kernel (see below)
%\usepackage[latest]{latexrelease} % will be used once available in more distributions (ISSUE #107)
\PassOptionsToPackage{printonlyused,smaller}{acronym} 
    \usepackage{acronym} % nice macros for handling all acronyms in the thesis
    %\renewcommand{\bflabel}[1]{{#1}\hfill} % fix the list of acronyms --> no longer working
    %\renewcommand*{\acsfont}[1]{\textsc{#1}} 
    \renewcommand*{\aclabelfont}[1]{\acsfont{#1}}
% ****************************************************************************************************


% ****************************************************************************************************
% 4. Setup floats: tables, (sub)figures, and captions
% ****************************************************************************************************
\usepackage{tabularx} % better tables
    \setlength{\extrarowheight}{3pt} % increase table row height
\newcommand{\tableheadline}[1]{\multicolumn{1}{c}{\spacedlowsmallcaps{#1}}}
\newcommand{\myfloatalign}{\centering} % to be used with each float for alignment
\usepackage{caption}
% Thanks to cgnieder and Claus Lahiri
% http://tex.stackexchange.com/questions/69349/spacedlowsmallcaps-in-caption-label
% [REMOVED DUE TO OTHER PROBLEMS, SEE ISSUE #82]    
%\DeclareCaptionLabelFormat{smallcaps}{\bothIfFirst{#1}{~}\MakeTextLowercase{\textsc{#2}}}
%\captionsetup{font=small,labelformat=smallcaps} % format=hang,
\captionsetup{font=small} % format=hang,
\usepackage{subfig}  
% ****************************************************************************************************


% ****************************************************************************************************
% 5. Setup code listings
% ****************************************************************************************************
% \usepackage{listings} 
% %\lstset{emph={trueIndex,root},emphstyle=\color{BlueViolet}}%\underbar} % for special keywords
% \lstset{language={Haskell},morekeywords={PassOptionsToPackage,selectlanguage},keywordstyle=\color{RoyalBlue},basicstyle=\small\ttfamily,commentstyle=\color{Green}\ttfamily,stringstyle=\rmfamily,numbers=none,numberstyle=\scriptsize,stepnumber=5,numbersep=8pt,showstringspaces=false,breaklines=true,belowcaptionskip=.75\baselineskip} 
% ****************************************************************************************************             


% ****************************************************************************************************
% 6. PDFLaTeX, hyperreferences and citation backreferences
% ****************************************************************************************************
% ********************************************************************
% Using PDFLaTeX
% ********************************************************************
\PassOptionsToPackage{pdftex,hyperfootnotes=false,pdfpagelabels}{hyperref}
    \usepackage{hyperref}  % backref linktocpage pagebackref
\pdfcompresslevel=9
\pdfadjustspacing=1 
\PassOptionsToPackage{pdftex}{graphicx}
    \usepackage{graphicx} 
 

% ********************************************************************
% Hyperreferences
% ********************************************************************
\hypersetup{%
    %draft, % = no hyperlinking at all (useful in b/w printouts)
    colorlinks=true, linktocpage=true, pdfstartpage=3, pdfstartview=FitV,%
    % uncomment the following line if you want to have black links (e.g., for printing)
    %colorlinks=false, linktocpage=false, pdfstartpage=3, pdfstartview=FitV, pdfborder={0 0 0},%
    breaklinks=true, pdfpagemode=UseNone, pageanchor=true, pdfpagemode=UseOutlines,%
    plainpages=false, bookmarksnumbered, bookmarksopen=true, bookmarksopenlevel=1,%
    hypertexnames=true, pdfhighlight=/O,%nesting=true,%frenchlinks,%
    urlcolor=webbrown, linkcolor=RoyalBlue, citecolor=webgreen, %pagecolor=RoyalBlue,%
    %urlcolor=Black, linkcolor=Black, citecolor=Black, %pagecolor=Black,%
    pdftitle={\myTitle},%
    pdfauthor={\textcopyright\ \myName, \myUni, \myFaculty},%
    pdfsubject={},%
    pdfkeywords={},%
    pdfcreator={pdfLaTeX},%
    pdfproducer={LaTeX with hyperref and classicthesis}%
}   

% ********************************************************************
% Setup autoreferences
% ********************************************************************
% There are some issues regarding autorefnames
% http://www.ureader.de/msg/136221647.aspx
% http://www.tex.ac.uk/cgi-bin/texfaq2html?label=latexwords
% you have to redefine the makros for the 
% language you use, e.g., american, ngerman
% (as chosen when loading babel/AtBeginDocument)
% ********************************************************************
\makeatletter
\@ifpackageloaded{babel}%
    {%
       \addto\extrasamerican{%
			\renewcommand*{\figureautorefname}{Figure}%
			\renewcommand*{\tableautorefname}{Table}%
			\renewcommand*{\partautorefname}{Part}%
			\renewcommand*{\chapterautorefname}{Chapter}%
			\renewcommand*{\sectionautorefname}{Section}%
			\renewcommand*{\subsectionautorefname}{Section}%
			\renewcommand*{\subsubsectionautorefname}{Section}%     
                }%
       \addto\extrasngerman{% 
			\renewcommand*{\paragraphautorefname}{Absatz}%
			\renewcommand*{\subparagraphautorefname}{Unterabsatz}%
			\renewcommand*{\footnoteautorefname}{Fu\"snote}%
			\renewcommand*{\FancyVerbLineautorefname}{Zeile}%
			\renewcommand*{\theoremautorefname}{Theorem}%
			\renewcommand*{\appendixautorefname}{Anhang}%
			\renewcommand*{\equationautorefname}{Gleichung}%        
			\renewcommand*{\itemautorefname}{Punkt}%
                }%  
            % Fix to getting autorefs for subfigures right (thanks to Belinda Vogt for changing the definition)
            \providecommand{\subfigureautorefname}{\figureautorefname}%             
    }{\relax}
\makeatother


% ****************************************************************************************************
% 7. Last calls before the bar closes
% ****************************************************************************************************
% ********************************************************************
% Development Stuff
% ********************************************************************
\listfiles
%\PassOptionsToPackage{l2tabu,orthodox,abort}{nag}
%   \usepackage{nag}
%\PassOptionsToPackage{warning, all}{onlyamsmath}
%   \usepackage{onlyamsmath}

% ********************************************************************
% Last, but not least...
% ********************************************************************
\usepackage{classicthesis} 
% ****************************************************************************************************


% ****************************************************************************************************
% 8. Further adjustments (experimental)
% ****************************************************************************************************
% ********************************************************************
% Changing the text area
% ********************************************************************
%\linespread{1.05} % a bit more for Palatino
%\areaset[current]{312pt}{761pt} % 686 (factor 2.2) + 33 head + 42 head \the\footskip
%\setlength{\marginparwidth}{7em}%
%\setlength{\marginparsep}{2em}%

% ********************************************************************
% Using different fonts
% ********************************************************************
%\usepackage[oldstylenums]{kpfonts} % oldstyle notextcomp
%\usepackage[osf]{libertine}
%\usepackage[light,condensed,math]{iwona}
%\renewcommand{\sfdefault}{iwona}
%\usepackage{lmodern} % <-- no osf support :-(
%\usepackage{cfr-lm} % 
%\usepackage[urw-garamond]{mathdesign} <-- no osf support :-(
%\usepackage[default,osfigures]{opensans} % scale=0.95 
%\usepackage[sfdefault]{FiraSans}
% ****************************************************************************************************
 % En classicthesis-config.tex se almacenan las opciones propias de la plantilla.

% Color institucional UGR
% \definecolor{ugrColor}{HTML}{ed1c3e} % Versión clara.
\definecolor{ugrColor}{HTML}{c6474b}  % Usado en el título.
\definecolor{ugrColor2}{HTML}{c6474b} % Usado en las secciones.

% Datos de portada
\usepackage{titling} % Facilita los datos de la portada
\author{Guillermo Galindo Ortuño}
\date{\today}
\title{A Sugiyama Like Decoding Algorithm}

% Portada
\usepackage{datetime}
\renewcommand\maketitle{
  \begin{titlepage}
    \begin{addmargin}[-2.5cm]{-3cm}
      \begin{center}
        \large  
        \hfill
        \vfill

        \begingroup
        \color{ugrColor}\spacedallcaps{\thetitle} \\ \bigskip
        \endgroup

        \spacedlowsmallcaps{\theauthor}

        \vfill

        Trabajo Fin de Grado \\ \medskip 
        Doble Grado en Ingeniería Informática y Matemáticas \\  \bigskip\bigskip


        \textbf{Tutores}\\
        André Weil \\ 
        Jean Dieudonné \\ \bigskip

        \spacedlowsmallcaps{Facultad de Ciencias} \\
        \spacedlowsmallcaps{E.T.S. Ingenierías Informática y de Telecomunicación} \\ \medskip
        
        \textit{Granada, a \today}

        \vfill                      

      \end{center}  
    \end{addmargin}       
  \end{titlepage}}
\usepackage{wallpaper}
\usepackage[main=spanish]{babel}

\begin{document}

\ThisULCornerWallPaper{1}{ugrA4.pdf}
\maketitle
\tableofcontents

\chapter*{Resumen}

\cite{turing1936a}
% % Los artículos y libros incluidos en el archivo research.bib pueden
% % citarse desde cualquier punto del texto usando ~\cite.

% Nos basamos en el trabajo desarrollado en~\cite{turing1936a}.

% Occaecati expedita cumque est. Aut odit vel nobis praesentium dolorem
% sed eligendi. Inventore molestiae delectus voluptatibus
% consequatur. Et cumque quia recusandae fugiat earum repellat
% porro. Earum et tempora vel voluptas. At sed animi qui hic eaque
% velit.

% Saepe deleniti aut voluptatem libero dolores illum iusto
% iusto. Explicabo dolor quia id enim molestiae praesentium sit. Odit
% enim doloribus aut assumenda recusandae. Eligendi officia nihil
% itaque. Quas fugiat aliquid qui est.

% Quis amet sint enim. Voluptatem optio quia voluptatem. Perspiciatis
% molestiae ut laboriosam repudiandae nihil.

% \ctparttext{
%   \color{black}
%   \begin{center}
%     Esta es una descripción de la parte de matemáticas.
%     Nótese que debe escribirse antes del título
%   \end{center}
% }
% \part{Parte de matemáticas}

% \chapter{Sección primera}

% Lorem ipsum dolor sit amet, consectetur adipiscing elit. Nunc finibus
% augue a tellus volutpat, quis cursus mi egestas. Suspendisse cursus
% eleifend lacus non consectetur. Proin nibh nisl, tincidunt eu
% tincidunt et, auctor vitae mi. Mauris tincidunt finibus enim, sit amet
% facilisis mi imperdiet in. Pellentesque pellentesque felis sed
% condimentum congue. Nunc at mauris et velit tempor gravida faucibus
% sed velit. Sed risus justo, feugiat non accumsan vitae, commodo a
% mauris. Vestibulum ac tortor ligula. Donec pulvinar neque ac purus
% varius dapibus. Donec fermentum bibendum ultrices. Quisque dictum,
% purus quis semper lacinia, justo nibh cursus orci, vitae sagittis
% mauris ex aliquet odio. Nullam vel lacus eu tellus dictum tempus in
% vel sapien. Morbi pulvinar tincidunt tincidunt. Cras eros tortor,
% commodo at tempus a, ultrices vel magna. Nulla eget mauris
% arcu. Vivamus aliquet sem odio, non faucibus ante iaculis non.

% Aenean aliquet metus nisi, eu congue nulla egestas nec. Etiam pretium
% cursus orci a venenatis. Pellentesque placerat feugiat tortor tempus
% laoreet. Sed at rutrum leo. Aenean magna nulla, egestas quis efficitur
% a, consequat id ligula. Nulla facilisi. Duis orci urna, bibendum vel
% mollis non, commodo vitae tellus. Integer luctus ex sed nibh
% convallis, id malesuada metus facilisis.

% \begin{theorem}
% En las condiciones dadas anteriormente se tiene
% \[
% g(t) = \int^b_a K(t,s)f(s)\ \diff s.
% \]
% \end{theorem}

% Vivamus sit amet sodales elit, in tempus nulla. Nullam euismod rhoncus
% quam, ac fringilla massa fermentum vehicula. Maecenas vestibulum purus
% non pellentesque congue. Sed nec massa purus. Mauris et quam
% dignissim, mollis lacus id, tristique diam. Vestibulum molestie
% vehicula tellus quis dignissim. Curabitur accumsan augue ut dictum
% semper. Proin tristique quam nulla, vel placerat tortor feugiat
% vel. Vestibulum quis ultricies enim, in vestibulum ante.

% \chapter{Sección segunda}
% Vivamus fringilla egestas nulla ac lobortis. Etiam viverra est risus,
% in fermentum nibh euismod quis. Vivamus suscipit arcu sed quam dictum
% suscipit. Maecenas pulvinar massa pulvinar fermentum
% pellentesque. Morbi eleifend nec velit ut suscipit. Nam vitae
% vestibulum dui, vel mollis dolor. Integer quis nibh sapien.

% Nullam laoreet augue at erat consectetur fermentum. In interdum
% aliquam condimentum. Etiam luctus viverra tellus, a consequat justo
% venenatis id. Curabitur a scelerisque erat. Nulla et ante eget lacus
% varius accumsan. Suspendisse odio nisl, convallis ac orci nec,
% efficitur semper enim. Pellentesque eros ipsum, luctus et viverra
% maximus, auctor at est. In hac habitasse platea dictumst. Etiam
% ultricies suscipit diam, eget dignissim tellus. Proin justo orci,
% volutpat placerat libero a, gravida interdum nisi.

% \begin{corollary}
% Se obtiene el siguiente resultado
% \[\begin{tikzcd}
% a\rar{f} \dar[swap]{g} & b \dar{h} \\
% c\rar{k} & d
% \end{tikzcd}\]
% \end{corollary}
% \begin{proof}
%   Vivamus sit amet sodales elit, in tempus nulla. Nullam euismod
%   rhoncus quam, ac fringilla massa fermentum vehicula. Maecenas
%   vestibulum purus non pellentesque congue. Sed nec massa purus.
%   Quod erat demostrandum.
% \end{proof}

% \ctparttext{\color{black}\begin{center}
% Esta es una descripción de la parte de informática.
% \end{center}}

% \part{Parte de informática}
% \chapter{Sección tercera}
% El siguiente código es un ejemplo de coloreado de sintaxis e inclusión
% directa de código fuente en el texto usando \texttt{minted}.

% \begin{minted}[frame=lines]{haskell}
% -- From the GHC.Base library.
% class  Functor f  where
%     fmap        :: (a -> b) -> f a -> f b

%     -- | Replace all locations in the input with the same value.
%     -- The default definition is @'fmap' . 'const'@, but this may be
%     -- overridden with a more efficient version.
%     (<$)        :: a -> f b -> f a
%     (<$)        =  fmap . const

% -- | A variant of '<*>' with the arguments reversed.
% (<**>) :: Applicative f => f a -> f (a -> b) -> f b
% (<**>) = liftA2 (\a f -> f a)
% -- Don't use \$ here, see the note at the top of the page

% -- | Lift a function to actions.
% -- This function may be used as a value for `fmap` in a `Functor` instance.
% liftA :: Applicative f => (a -> b) -> f a -> f b
% liftA f a = pure f <*> a
% -- Caution: since this may be used for `fmap`, we can't use the obvious
% -- definition of liftA = fmap.

% -- | Lift a ternary function to actions.
% liftA3 :: Applicative f => (a -> b -> c -> d) -> f a -> f b -> f c -> f d
% liftA3 f a b c = liftA2 f a b <*> c

% {-# INLINABLE liftA #-}
% {-# SPECIALISE liftA :: (a1->r) -> IO a1 -> IO r #-}
% {-# SPECIALISE liftA :: (a1->r) -> Maybe a1 -> Maybe r #-}
% {-# INLINABLE liftA3 #-}
% {-# SPECIALISE liftA3 :: (a1->a2->a3->r) -> IO a1 -> IO a2 -> IO a3 -> IO r #-}
% {-# SPECIALISE liftA3 :: (a1->a2->a3->r) ->
%                                 Maybe a1 -> Maybe a2 -> Maybe a3 -> Maybe r #-}

% -- | The 'join' function is the conventional monad join operator. It
% -- is used to remove one level of monadic structure, projecting its
% -- bound argument into the outer level.
% join              :: (Monad m) => m (m a) -> m a
% join x            =  x >>= id
% \end{minted}

% Vivamus fringilla egestas nulla ac lobortis. Etiam viverra est risus,
% in fermentum nibh euismod quis. Vivamus suscipit arcu sed quam dictum
% suscipit. Maecenas pulvinar massa pulvinar fermentum
% pellentesque. Morbi eleifend nec velit ut suscipit. Nam vitae
% vestibulum dui, vel mollis dolor. Integer quis nibh sapien.

\chapter{Conceptos básicos sobre códigos lineales}%
\label{chap:conceptos_básicos_sobre_códigos_lineales}

\section{Códigos lineales, matrices generatriz y de paridad}%
\label{sec:códigos_lineales_matrices_generatriz_y_de_paridad}

Sea \(\F_q^n\) el espacio vectorial de todas las \(n\)-tuplas sobre el cuerpo finito \(\F_q\).

\begin{definition}[Código]
 Definimos un \((n, M)\) \textit{código} \(\mathcal{C}\) sobre \(\F_q\) como un subconjunto de \(\F_q^n\) de tamaño \(M\).
\end{definition}

Normalmente escribiremos los vectores \((a_1, a_2, \dots, a_n)\) en \(\F_q^n\) de la forma \(a_1 a_2 \cdots a_n\) y llamaremos a los vectores en \(\mathcal{C}\) \textit{palabras código}, o simplemente \textit{palabras}. Además, utilizaremos nombres concretos para referirnos a códigos sobre algunos de los cuerpos más comunes . A los códigos sobre \(\F_2\) los llamaremos \textit{códigos binarios}, los códigos sobre \(\F_3\) los notaremos como \textit{códigos ternarios} y a los códigos sobre \(F_4\) los llamaremos \textit{códigos cuaternarios}.\\

Dicho esto, necesitamos dotar de algún tipo de estructura adicional a los códigos pues siendo únicamente conjuntos estamos muy limitados. La estructura adicional más útil que se les impone es la linealidad.

\begin{definition}[Código lineal]
Si \(\mathcal{C}\) es un subespacio vectorial de dimensión \(k\) de \(\F_q^n\), diremos que \(\mathcal{C}\) es un \([n,k]\) \textit{código lineal} sobre \(\F_q\).
\end{definition}

Dicho esto, las dos maneras más comunes de presentar un código lineal son dando una matriz generatriz o una matriz de paridad.

\begin{definition}[Matriz generatriz]
    Una \textit{matriz generatriz} de un \([n,k]\) código lineal \(\mathcal{C}\) es cualquier matriz \(G\) de dimensiones \(k \times n\) cuyas filas formen una base de \(\mathcal{C}\).
\end{definition}

Dado una matriz generatriz \(G\), para cualquier conjunto de \(k\) columnas independientes de esta, diremos que el correspondiente conjunto de coordenadas es un \textit{conjunto de información} de C. Las restantes \(r=n-k\) coordenadas las notaremos como \textit{conjunto de redundancia}, y llamaremos a \(r\) \textit{redundancia} de \(\C\). Si las primeras \(k\) coordenadas forman un conjunto de información, existe una única matriz generatriz para el código de la forma \([I_k | A]\) donde \(I_k\) es la matriz identidad de orden \(k\). Diremos que una matriz generatriz así está en \textit{forma estándar}.

\begin{definition}
    Una \textit{matriz de paridad } de un \([n,k]\) código lineal \(\C\) es cualquier matriz \(H\) de dimensiones \((n-k)\times n\) tal que
    \[
    \C = \{x \in \F_q^n | Hx^T = 0\}
    .\]
\end{definition}

Como un código lineal es un subespacio de un espacio vectorial, es el núcleo de alguna aplicación lineal, y por tanto, para un código lineal siempre existe alguna matriz de paridad \(H\). Mencionemos que las filas de \(H\) son también independientes. Esto es porque, al ser \(H\) una aplicación lineal de \(\F_q^n\) en  \(\F_q^{n-k}\), y la dimensión del núcleo de dicha aplicación es \(k\), tenemos que la dimensión de la imagen es \(n-k\) y por tanto el rango de \(H\) también.

\section{Códigos duales}%
\label{sec:códigos_duales}

Como mencionamos en la sección anterior, las filas de una matriz de paridad de un código \(\C\) son independientes, y por tanto \(H\) es a su vez la matriz generatriz de otro código, el llamado \textit{dual} o \textit{ortogonal} de \(\C\), denotado por \(\C^\bot\). Es inmediato ver que \(\C^\bot\) es un \([n, n-k]\) código. Además, es posible dar una construcción alternativa para \(\C^\bot\) utilizando el producto escalar en  \(\F_q^n\):
\[
\C^\bot = \{x \in \F_q^n | x \cdot c = 0\ \forall c \in \C\}
.\]

\section{Pesos y distancias}
Una de las características más útiles para el estudio de códigos es la distancia mínima entre las palabras de este.

\begin{definition}
Dados dos vectores \(x, y \in F_q^n\), definimos la distancia \textit{Hamming} entre ellos \(d(x,y)\) como el numero de coordenadas en las que  \(x\) e \(y\) difieren.
\end{definition}

Veamos que en efecto esta es una distancia:

\begin{proposition}
La función distancia \(d(x,y)\) satisface las siguientes condiciones:
\begin{nlist}
    \item \(d(x,y) \geq 0\) para todo \(x, y \in \F_q^n\).
    \item \(d(x,y) = 0\) si y solo si \(x = y\).
    \item \(d(x,y) = d(y,x)\) for all \(x,y \in \F_q^n\)
    \item \(d(x,z) \leq d(x,y) + d(y,z)\) para todo \(x, y, z \in \F_q^n\)
\end{nlist}

\begin{proofs}
Las tres primeras propiedades son obvias por la propia definición de la distancia. Veamos pues la propiedad iv).\\

Dados dos vectores \(x, y \in \F_q^n\), definimos el conjunto \(D(x,y) = \{i | x_i \neq y_i\}\), y denotamos el complementario por \(D^c(x, y) = \{i | x_i = y_i\}\). Es claro que entonces el cardinal de \(D(x,y)\) coincide con nuestra distancia. \\

Recordemos también algunas propiedades sobre cardinales de conjuntos. Sea un conjunto \(A\), notemos por \(|A|\) su cardinal. Entonces, para cualesquiera conjuntos \(A\) y \(B\):
\begin{nlist}
    \item \(|A| \leq |A \cup B|\)
    \item \(|A \cup |B| \leq |A| + |B|\)
    \item Si \(|A| \leq |B|\), entonces \(|A^c| \geq |B^c|\)
\end{nlist}
Con esto, dados \(x, y, z \in \F_q^n\), conjuntos tenemos que

\[
D^c(x,z) = \{i | x_i = z_i\} = \{i | x_i = z_i = y_i\} \cup \{i | x_i = z_i \neq y_i\}
\]
\[
\implies |D^c(x,z)| \geq |\{i | x_i = z_i = y_i\}| = |\{i | x_i = y_i\} \cap \{i | z_i = y_i\}|
\]
\[
\implies |D(x,z)| \leq |\{i | x_i \neq y_i\} \cup \{i | z_i \neq y_i\}| = |D(x,y) \cup D(y,z)|
\]
\[
\implies |D(x,z)| \leq |D(x,y)| + |D(y,z)|
\]
quedándo demostrado el resultado.
\end{proofs}
\end{proposition}

Ahora, la \textit{distancia (mínima)} de un código \(\C\) es la mínima distancia entre dos palabras distintas de dicho código. Esta propiedad será crucial a la hora de determinar el número de errores que podrá corregir un código.

\begin{definition}
Diremos que el peso (\textit{de Hamming}) \(wt(x)\) de un vector \(x \in \F_q^n\) es el número de coordenadas distintas de cero de \(x\).
\end{definition}

Si tenemos dos vectores \(x, y \in \F_q^n\) es inmediato comprobar que \(d(x,y) = wt(x - y)\). En el siguiente mostramos la relación entre la distancia y el peso.

\begin{proposition}
Si \(\C\) es un  código lineal, la distancia mínima coincide con el mínimo de los pesos de las palabras distintas de cero de C.

\begin{proofs}
Sea \(d = d(x,y)\) la distancia mínima del código \(\C\), que se alcanza entre dos vectores \(x, y \in \C\), y \(d' = wt(z)\) el peso mínimo que se alcanza en \(z \in C\).

Por ser \(\C\) un subespacio vectorial,  \(0 \in \C\), y por tanto  \(d \leq d(z, 0) = wt(z) = d'\). De nuevo por ser \(\C\) un subespacio vectorial tenemos que \(x-y \in \C\), y por tanto  \(d' \leq wt(x-y) = d(x,y) = d\), y por tanto \(d = d'\).
\end{proofs}
\end{proposition}

Como consecuencia a este resultado, para códigos lineales, a la distancia mínima también se le llama \textit{peso mínimo} del código. En adelante, si el peso mínimo \(d\) de un \([n,k]\) código es conocido, entonces nos referiremos al código como un \([n,k,d]\) código.

\section{Codificar, decodificar, y el Teorema de Shannon}

\subsection{Codificar}%
Sea \(\C\) un \([n,k]\) código lineal sobre el cuerpo \(\F_q\) con matriz generatriz \(G\). Como este es un subespacio vectorial de \(\F_q^n\) de dimensión \(k\), contiene \(q^k\) palabras, que están en correspondencia uno a uno con \(q^k\) posibles mensajes. Por esto, la forma más simple es ver estos mensajes como \(k\)-tuplas \(x\) en \(\F_q^k\). Así, lo más común es codificar un mensaje \(x\) como la palabra \(c = xG\). Si G está en forma estándar, las primeras \(k\) coordenadas son los símbolos de información \(x\); el resto de \(n-k\) símbolos son los simbolos de paridad, es decir, la redundancia añadida a \(x\) con el fin de poder recuperarla si ocurre algún error. Dicho esto, la matriz \(G\) puede no estar en forma estándar. En particular, si existen índices de columnas \(i_1, i_2, \dots, i_n \) tales que la matriz \(k \times k\) formada por estas columnas es la matriz identidad, entonces el mensaje se encuentra en las coordenadas \(i_1, i_2, \dots, i_n \) separado pero sin modificar, es decir, el símbolo del mensaje \(x_j\) se encuentra en la componente \(i_j\) de la palabra código. Si esto ocurre diremos que el codificador e s \textit{sistemático}.

\subsection{Decodificar y el Teorema de Shannon}%
\label{sub:decodificar_y_el_teorema_de_shannon}

El proceso de decodificar, consistente en determinar qué palabra (y por tanto qué mensaje \(x\)) fue mandado al recibir un vector \(y\), es más complejo. Encontrar algoritmos de decodificación eficientes es una área de investigación muy relevante en la teoría de códigos debido a sus aplicaciones prácticas. En general, codificar es sencillo y decodificar es complicado, especialmente si tiene un tamaño suficientemente grande.

\chapter{Introducción a extensiones de Ore}%
\label{chap:conceptos_básicos_sobre_extensiones_ore}

A continuación introduciremos los conceptos principales sobre extensiones de Ore que utilizaremos como base para el siguiente capítulo. Las definiciones y principales resultados de este capítulo siguen el desarrollo realizado en~\cite{factoring_ore}.

\section{Conceptos básicos sobre extensiones de Ore}

Aunque las definiciones que realizamos a continuación se pueden hacer más generales, en este caso nos restringiremos al caso más particular necesario para continuar. Dicho esto, nuestro algoritmo trabajará sobre polinomios de Ore no conmutativos, con una única indeterminada \(x\), con coeficientes en un cuerpo cualquiera. Precisando, nuestros polinomios serán elementos de un anillo asociativo \(R = \F[x;\sigma]\), donde

\begin{itemize}
    \item \(\F\) es un cuerpo cualquiera.
    \item \(\sigma: \F \to \F\) es un automorfismo de cuerpos de orden finito digamos \(n\).
\end{itemize}

La construcción de \(R = \F[x;\sigma]\) sigue de la siguiente forma:
\begin{itemize}
    \item \(R\) es un \(\F\)-espacio vectorial a la izquierda sobre la base  \(\{x^n: n \geq 0\}\). Entonces, los elementos de  \(R\) son polinomios a la izquierda de la forma \(a_0 + a_1 x + \cdots + a_n x^n\) con \(a_i \in \F\).
    \item La suma de polimios es la usual.
    \item El producto de \(R\) está basado en las siguientes reglas: \(x^n x^m = x^{n+m}\), para \(m, n \in \mathbb{N}\), y \(xa = \sigma(a)x\) para  \(a \in \F\). Este producto se extiende recursivamente a \(R\).
\end{itemize}

Ahora que hemos definido formalmente las extensiones de Ore, introduciremos varios conceptos análogos a los anillos de polinomios conmutativos comunes.

El grado \(\deg (f)\) de un polinomio no nulo \(f \in R\), al igual que su coeficiente líder, se definen de la manera usual, de forma que
\[
f = \lc(f)x^{\deg(f)} + f_{\downarrow}
,\]
donde \(f_{\downarrow}\) es un polinomio de grado menor que \(\deg(f)\), y el coeficiente líder  \(\operatorname{lc}(f)\) es no nulo. Siguiendo las convenciones usuales para el polinomio nulo diremos que \(\deg(0) = -\infty\), y que \(\lc(0) = 0\).

Al igual que ocurre para polinomios conmutativos, directamente de la definición del producto de polinomios, dados \(f, g \in R\):
\[
\deg(fg) = \deg(f) + \deg(g)
.\]

Además, de la misma definición obtenemos también que
\[
\lc(fg) = \lc(f)\sigma^{\deg(f)}(\lc(g))
,\]

y por tanto \(R\) es un dominio de integridad no conmutativo.

El anillo \(R\) tiene algoritmos de división a la izquierda y derecha (veánse los algoritmos~\ref{alg:left_euclidean_div} y~\ref{alg:right_euclidean_div}).

\begin{algorithm}[H]
 \label{alg:left_euclidean_div}
 \SetKwInput{KwIn}{Entrada}
 \SetKwInput{KwOut}{Salida}
 \SetKw{Initialization}{Inicialización:}
 \KwIn{\(f,g \in \F[x;\sigma] \text{ con } g \neq 0\)}
 \KwOut{\(q,r \in \F[x;\sigma] \text{ tales que } f = qg + r \text{ y } \deg(r) < \deg(g)\)}
 \Initialization{\(q:=0, r:=f\)} \\
 \While{\(\deg(g) \leq \deg(r)\)}{
  \(a = \lc(r)\sigma^{deg(r) - deg(g)}(\lc(g)^{-1})\) \\
  \(q := q + ax^{\deg(r) - \deg(g)}, r:= r - ax^{deg(r) - deg(g)}g\)
 }
 \caption{División Euclídea a la izquierda}
\end{algorithm}

\begin{algorithm}[H]
 \label{alg:right_euclidean_div}
 \SetKwInput{KwIn}{Entrada}
 \SetKwInput{KwOut}{Salida}
 \SetKw{Initialization}{Inicialización:}
 \KwIn{\(f,g \in \F[x;\sigma] \text{ con } g \neq 0\)}
 \KwOut{\(q,r \in \F[x;\sigma] \text{ tales que } f = gq + r \text{ y } \deg(r) < \deg(g)\)}
 \Initialization{\(q:=0, r:=f\)} \\
 \While{\(\deg(g) \leq \deg(r)\)}{
  \(a = \sigma^{-deg(g)}(\lc(g)^{-1}\lc(r))\) \\
  \(q := q + ax^{\deg(r) - \deg(g)}, r:= r - gax^{deg(r) - deg(g)}\)
 }
 \caption{División Euclídea a la derecha}
\end{algorithm}

Mostramos a continuación una demostración que justifica estos algoritmos, cuya versión original puede encontrarse en~\cite[Th. 4.34]{bueso2003algorithmic}

\begin{theorem}
Sea \(\F\) un cuerpo finito de \(q\) elementos siendo \(q\) una potencia de un primo,  \(\sigma\) un autormorfismo de \(\F\) no nulo, y \(R = \F[x;\sigma]\) la extensión de Ore correspondiente. Entonces, dados \(f, g \in R\) existen \(q, r \in R\) únicos tales que:
\begin{enumerate}
    \item \(f = qg + r\).
    \item \(\deg(r) < \deg(g)\).
\end{enumerate}

Bajo las mismas hipótesis, existen también \(q, r \in R\) únicos tales que:
\begin{enumerate}
    \item \(f = gq + r\)
    \item \(\deg(r) < \deg(g)\).
\end{enumerate}
\end{theorem}

\begin{proof}
Para abreviar digamos \(m = \deg(g), n = \deg(f)\). Veamos primero la prueba de la división a la izquierda. Si \(m > n\), entonces no tenemos nada que probar, pues tomando  \(q = 0, r = f\) se cumple el resultado. Por otro lado, si \(m \leq n\), sean \(f = \sum_{i=0}^{n} a_i x^i\) y \(g = \sum_{j=0}^{m} b_j x^j\), aplicaremos inducción sobre \(n\). Si \(n = 0\), entonces también \(m = 0\), así que  \(f = a_0, g = b_0\), y por tanto tomamos \(r = 0, q = a_0 b_0^{-1}\).

Por tanto, supongamos la afirmación cierta para todo \(f\) de grado menor que \(n\).  Sea \(a = a_n \sigma^{n-m}(b_m^{-1})\). Entonces es claro que
\[
\deg(ax^{n-m}g) = n,
\]
\[
\lc(ax^{n-m}g) = a_n
.\]
Por tanto tenemos que
\[
\deg(f - a x^{n-m}g) < n,
\]
y por tanto, la hipótesis de inducción nos dice que existen \(q'\) y \(r'\) compliendo que \(\deg(r') < \deg(g)\)  y
\[
f - a x^{n-m}g = q'g + r'
.\]

Sea
\[
q = a x^{n-m} + q'
,\]

entonces
\[
f = a x^{n-m}g + q'g + r' = qg + r'
.\]

Queda probar que \(q\) y \(r\) son únicos como tales. Supongamos que
\[
f = q_1g + r_1 = q_2g + r_2,
\]
con \(\deg(r_1), \deg(r_2) < \deg(g)\). Entonces, \((q_1 - q_2)g = r_2 - r_1\), y podemos afirmar entonces que
\[
\deg(q_1 - q_2) + \deg(g) = \deg((q_1-q_2)g)
\]
\[
= \deg(r_2-r_1) \leq \max(\deg(r_2), \deg(r_1)) < \deg(g)
.\]
Esto prueba que \(\deg(q_1-q_2) = -\infty\), demostrando que \(q_1 - q_2 = 0\) y que \(r_2 = r_1 = 0\) que termina la prueba de la primera parte.

La prueba de la división a la derecha es completamente análoga, tomando \(a = \sigma^{-m}(a_{n}b_{m}^{-1})\), y utilizando que \(\deg(f - gax^{n-m}) < n\).

\end{proof}

Los polinomios \(r\) y \(q\) obtenidos como salida del algoritmo \ref{alg:left_euclidean_div} los llamaremos \textit{resto a la izquierda} y  \textit{cociente a la izquierda}, respectivamente, de la división a la izquierda de \(f\) por \(g\). Utilizaremos la notación \(r = \operatorname{lrem}(f,g)\) y \(q = \operatorname{lquo}(f,g)\). Asumimos convenciones y notaciones análogas para el algoritmo de división a la derecha.

Un polinomio \(f \in R\) se dice \textit{central} si para cualquier otro polinomio \(g \in R\), se tiene que \(fg = gf\).

\begin{lemma}
\label{lem:central_decomposition}
    Sea \(f \in R\) un polinomio central y \(g,h \in R\) tales que \(f = gh\). Entonces también se cumple la igualdad \(f = hg\)
\end{lemma}

\begin{proof}
    Multiplicando \(f\) a la derecha por \(g\) y usando que \(f\) es central tenemos que, \( fg = ghg = ggh\), y por tanto \(hg = gh = f.\)
\end{proof}

\section{Máximo común divisor y mínimo común multiplo}%
\label{sec:máximo_común_divisor_y_mínimo_común_multiplo}

Veamos ahora que, como consecuencia del algoritmo de división a la izquierda, dado un ideal a la izquierda \(I\) de \(R\), y cualquier polinomio no nulo  \(f \in I\) de grado mínimo, \(f\) es un generador de  \(I\). Notaremos en este caso \(I = Rf\).

En efecto, para cualquier \(g \in I\), utilizando el algoritmo~\ref{alg:left_euclidean_div} tenemos que \(g = q*f + r\) con \(\deg(r) < \deg(f)\). Pero \(g\) y \(q*f\) pertenecen a \(I\), y por tanto  \(r\) también. Como \(f\) era de grado mínimo en \(I\),  \(r = 0\).

Análogamente se prueba que cualquier ideal a la derecha de \(R\) es principal.  Por tanto \(R\) es un dominio de ideales principales no conmutativo.

Dados \(f,g \in R\), \(Rf \subset Rg\) significa que \(g\) es un \textit{divisor a la derecha} de \(f\), simbólicamente \(g|_{r} f\), o que \(f\) es \textit{múltiplo a la izquierda} de \(g\).

Por ser \(R\) un dominio de ideales principales sabemos que \(Rf + Rg = Rd\) para algún \(d \in R\), y es inmediato comprobar que \(d |_r f\) y  \(d |_r g\). Además, si tenemos \(d'\) con \(d' |_r f\), \(d' |_r g\), entonces \(Rf + Rg \subset Rd'\), luego \(Rd \subset Rd'\) y por tanto  \(d |_r d'\). En este caso diremos que \(d\) es un \textit{máximo común divisor a la derecha} de \(f\) y \(g\), estando unívocamente determinado salvo múltiplicación a la izquierda por una unidad de \(R\). Utilizaremos la notación \(d = {(f,g)}_r\). Además de aquí obtenemos directamente la \textbf{identidad de Bezout}.

\begin{proposition}[Identidad de bezout]
Sean \(f \text{ y } g\) dos polinomios en \(R\), y  \(d = {( f,g )_r}\), entonces existen \(u, v \in R\) tales que
\[
uf + vg = d
.\]
\end{proposition}

Similarmente  \(Rf \cap Rg = Rm\) si y solo si \(m\) es un \textit{mínimo común múltiplo a la izquierda} de \(f\) y \(g\), notado por \(m = {[f,g]}_l\). Este también es único salvo multiplicación a la izquierda por una unidad de \(R\).

La versión a la derecha de todas estas definiciones y propiedades puede establecerse de forma completamente análoga.

A continuación mostramos las respectivas versiones del Algoritmo Extendido de Euclides a derecha e izquierda (algoritmos~\ref{alg:right_ext_euc_alg} y~\ref{alg:left_ext_euc_alg} respectivamente). Estas nos permiten calcular el máximo común divisor y el mínimo común multiplo tanto a izquierda como a derecha.
En particular, para nuestro algoritmo principal utilizaremos la versión a la derecha de este algoritmo para obtener los coeficientes de Bezout.

\begin{algorithm}[H]\label{alg:right_ext_euc_alg}
 \SetKwInput{KwIn}{Entrada}
 \SetKwInput{KwOut}{Salida}
 \SetKw{Initialization}{Inicialización:}
 \KwIn{\(f,g \in \F[x;\sigma] \text{ con } f \neq 0,\ g \neq 0\)}
 \KwOut{\(\{u_i, v_i, r_i\}_{i = 0, \dots, h, h+1} \text{ tales que } r_i = fu_i + gv_i \text{ para todo  }\) \(0 \le i \le h+1, r_h = (f,g)_l, \text{ y } fu_{h+1} = [f,g]_r.\)}
 \Initialization{\\
 \(r_0 \leftarrow f, r_1 \leftarrow g\) \\
 \(u_0 \leftarrow 1, u_1 \leftarrow 0\) \\
 \(v_0 \leftarrow 0, v_1 \leftarrow 1\) \\
 \(q \leftarrow 0, rem \leftarrow 0\) \\
 \(i \leftarrow 1\) \\
 }
 \While{\(r_i \neq 0\)}{
  \(q,\ rem \leftarrow \operatorname{rquot-rem}(r_{i-1}, r_i)\) \\
  \(r_{i+1} \leftarrow rem\) \\
  \(u_{i+1} \leftarrow u_{i-1} - u_i q\) \\
  \(v_{i+1} \leftarrow v_{i-1} - v_{i} q\) \\
  \(i \leftarrow i +1\) \\
 \Return{\(\{u_i, v_i, r_i\}_{i = 0, \dots, h, h+1}\)}
 }
 \caption{Algoritmo extendido de Euclides a la derecha}
\end{algorithm}

\begin{algorithm}[H]\label{alg:left_ext_euc_alg}
 \SetKwInput{KwIn}{Entrada}
 \SetKwInput{KwOut}{Salida}
 \SetKw{Initialization}{Inicialización:}
 \KwIn{\(f,g \in \F[x;\sigma] \text{ con } f \neq 0,\ g \neq 0\)}
 \KwOut{\(\{u_i, v_i, r_i\}_{i = 0, \dots, h, h+1} \text{ tales que } r_i = u_i f + v_i g \text{ para todo  }\) \(0 \le i \le h+1, r_h = (f,g)_r, \text{ y } fu_{h+1} = [f,g]_l.\)}
 \Initialization{\\
 \(r_0 \leftarrow f, r_1 \leftarrow g\) \\
 \(u_0 \leftarrow 1, u_1 \leftarrow 0\) \\
 \(v_0 \leftarrow 0, v_1 \leftarrow 1\) \\
 \(q \leftarrow 0, rem \leftarrow 0\) \\
 \(i \leftarrow 1\) \\
 }
 \While{\(r_i \neq 0\)}{
  \(q,\ rem \leftarrow \operatorname{rquot-rem}(r_{i-1}, r_i)\) \\
  \(r_{i+1} \leftarrow rem\) \\
  \(u_{i+1} \leftarrow u_{i-1} - u_i q\) \\
  \(v_{i+1} \leftarrow v_{i-1} - v_{i} q\) \\
  \(i \leftarrow i +1\) \\
 \Return{\(\{u_i, v_i, r_i\}_{i = 0, \dots, h, h+1}\)}
 }
 \caption{Algoritmo extendido de Euclides a la izquierda}
\end{algorithm}

La prueba del teorema sobre estos algoritmos es una adaptación de la demostración del teorema original para dominios euclideos conmutativos que se encuentra en~\cite{algI}, a excepción del último resultado (el que calcula el mínimo común múltiplo) cuya demostración se encuentra en~\cite{bueso2003algorithmic}.
\begin{theorem}
    Los algoritmos~\ref{alg:right_ext_euc_alg} y~\ref{alg:left_ext_euc_alg} son correctos.
\end{theorem}

\begin{proof}
    Realizaremos únicamente la demostración para el algoritmo~\ref{alg:right_ext_euc_alg}, pues la demostración de la versión a la izquierda es análoga.

    En primer lugar mencionemos que siempre que \(r_i \neq 0\) se tiene que \(\deg(r_{i+1}) < \deg(r_i)\), por tanto existe  \(h \geq 1\) tal que \(r_h \neq 0\) pero \(r_{h+1} = 0\).

Para cada \(i \leq h\) tenemos que \(r_i \neq 0\), y por tanto podemos utilizar la división a la derecha de \(r_{i-1}\) entre \(r_i\) para obtener

\[
r_{i-1} = r_i q_{i+1} + r_{i+1}
.\]

De aquí obtenemos que los divisores a la izquierda comunes de \(r_{i-1}\) y de \(r_i\) coinciden con los divisores a la izquierda comunes de  \(r_i\) y de  \(r_{i+1}\). Luego
\[
r_h = (0, r_h)_l = (r_{h+1}, r_h)_l = (r_h, r_{h-1-})_l = \cdots = (r_1, r_0)_l = (g,f)_l
.\]

A continuación vamos a definir \(u_i, v_i \in R\) con \(i = 0,1, \dots, h, h+1\). En primer lugar, tomamos
\[
u_0 = 1,\ v_0 = 0,\ u_1 = 0,\ v_1 = 1
.\]
Entonces, para \(1 \leq i \leq h\) definimos
\[
u_{i+1} = u_{i-1} - u_iq_{i+1}, \quad
v_{i+1} = v_{i-1} - v_iq_{i+1}
.\]

Aplicaremos un argumento por inducción para comprobar que \(r_i = fu_i + gv_i\). Es inmediato comprobar que para \(i = 0,1\) se cumple, así que supongamos que se cumple para \(i-1,\ i\) y veamos que se cumple para \(i+1\). En efecto
\[
\begin{aligned}
f u_{i+1} + g v_{i+1} =& f(u_{i-1} - u_iq_{i+1}) + g(v_{i-1} - v_iq_{i+1}) = \\
&f u_{i-1} + g v_{i-1} - (f u_i + g v_i)q_{i+1} = r_{i-1} - r_iq_{i+1} = r_{i+1}
\end{aligned}
.\]

Para concluir veamos que \(u_{h+1}f = [f,g]_r\). Observemos en primer lugar que \(0 = r_{h+1} = fu_{h+1} + gv_{h+1}\), luego \(fu_{h+1} = -gv_{h+1}\) es un múltiplo a la derecha común de \(f\) y \(g\). Supongamos \(fu' = -gv'\) un múltiplo común a la derecha de \(f\) y \(g\) cualquiera. Entonces, definimos
\[
\begin{aligned}
    c_0 &= -v' \\
    c_1 &= u' \\
    &\ldots \\
    c_{i+1} &= c_{i-1} - q_{i+1}c_i\quad \text{para}\quad 1 \le i \le h  \\
\end{aligned}
.\]
Así, las siguientes igualdades se cumplen
\[
\begin{aligned}
    r_{i+1}c_{i} - r_{i}c_{i+1} &= 0 \\
    u_{i+1}c_{i} - u_{i}c_{i+1} &= (-1)^{i+1}u' \\
    v_{i+1}c_{i} - v_{i}c_{i+1} &= (-1)^{i+1}v' \\
\end{aligned}
.\]
para \(1 \le i \le h\). Para demostrarlas realizamos una inducción sencilla. Para \(i = 0\), tenemos que
\[
\begin{aligned}
    r_{1}c_{0} - r_{0}c_{1} &= -gv' - fu' =  0 \\
    u_{1}c_{0} - u_{0}c_{1} &= -c_{1} =  -u' \\
    v_{1}c_{0} - v_{0}c_{1} &= c_{0} = -v' \\
\end{aligned}
.\]

Supuestas las tres igualdades ciertas para \(i-1\),
 \[
\begin{aligned}
    r_{i+1}c_{i} - r_{i}c_{i+1} &= (r_{i-1} - r_{i}q_{i+1})c_i - r_i(c_{i-1} - q_{i+1}c_i) = 0 \\
    u_{i+1}c_{i} - u_{i}c_{i+1} &= (u_{i-1} - u_{i}q_{i+1})c_i - u_i(c_{i-1} - q_{i+1}c_i) = u_{i-1}c_i - u_ic_{i-1}= -((-1)^{i}u')  \\
    v_{i+1}c_{i} - v_{i}c_{i+1} &= (v_{i-1} - v_{i}q_{i+1})c_i - v_i(c_{i-1} - q_{i+1}c_i) = v_{i-1}c_i - v_ic_{i-1}= -((-1)^{i}v')  \\
\end{aligned}
.\]
Ahora, por \(r_{h+1} = 0 \neq r_{h}\), sabemos que \(c_{h+1} = 0\), y por tanto \(u_{h+1}\) y \(v_{h+1}\) dividen a la izquierda a \(u'\) y \(v'\) respectivamente. Luego,
\[
f u_{h+1} = -g v_{h+1} = {[f, g]_r}
.\]
\end{proof}

Veamos un lema que nos será útil posteriormente.

\begin{lemma}
\label{lem:reea}
    Sean \(f, g \in \F[x; \sigma] \) y \(  {\{u_{i}, v_{i}, r_{i}\}}_i = 0, \ldots, h\) los coeficientes obtenido al aplicar el Algoritmo Extendido de Euclides a la derecha a \(f\) y \(g\). Notemos \(R_{0} =
    \begin{pmatrix}
    u_0 & u_1 \\
    v_0 & v_1
    \end{pmatrix},\)
    \(Q_i =
    \begin{pmatrix}
        0 & 1 \\
        1 & -q_{i+1}
    \end{pmatrix}
    \)
    y \(R_i =  R_0 Q_1 \cdots Q_i\) para cualquier \(i = 0, \ldots, h\). Por tanto, para todo \(i = 0, \ldots, h\) se cumplen las siguientes afirmaciones:

    \begin{enumerate}
        \item \((fg)R_i = (r_{i} r_{i+1})\).
        \item \(R_i =
            \begin{pmatrix}
            u_i &  u_{i+1} \\
            v_{i} & v_{i+1}
            \end{pmatrix}\).
        \item \(R_i\) tiene inverso a izquierda y derecha.
        \item \((u_i, v_i)_r = 1\).
        \item \(\deg f = \deg r_{i-1} + \deg v_{i}\).

    \end{enumerate}
\end{lemma}
\begin{proof}
Veamos primero las dos primeras afirmaciones, pues probando 2 la afirmación 1 es inmediata por la demostración del algoritmo~\ref{alg:right_ext_euc_alg}. Razonando por inducción, la propiedad es cierta para \(i = 0\), así que supongamos que fijo \(i\) lo es para  \(i-1\). Entonces tenemos que
\[
R_i =
\begin{pmatrix}
    u_{i-1} & u_{i} \\
    v_{i-1} & v_{i}
\end{pmatrix}
\cdot
\begin{pmatrix}
    0 & 1 \\
    1 & -q_{i+1}
\end{pmatrix}
=
\begin{pmatrix}
    u_i & u_{i-1} - u_{i}q_{i+1} \\
    v_i & v_{i-1} - v_{i}q_{i+1} \\
\end{pmatrix}
=
\begin{pmatrix}
    u_{i} & u_{i+1} \\
    v_{i} & v_{i+1}
\end{pmatrix}
,\]

quedando demostrados las afirmaciones 1 y 2. De nuevo si probamos 3 obtenemos 4 inmediatamente. Observemos que \[T_i =
\begin{pmatrix}
q_{i+1} & 1 \\
1 & 0
\end{pmatrix} \]
es una inversa a la izquierda y derecha de \(Q_i\). Por tanto, \(S_i = T_i \cdots T_1 R_0\) es una inversa a izquierda y derecha de \(R_i\) y 3 y 4 quedan probadas.

Veamos 5. Para \(i = 1\),  \(r_0 = f\) y  \(v_1 = 1\) cumpliéndose la igualdad, así que razonemos por inducción. Mencionemos que \(\deg r_i < \deg r_{i-1}\) y que \(\deg v_{i-1} < \deg v_i\) para cualquier \(i > 1\). Entonces, como \(r_{i+1} = r_{i-1} - r_{i}q_{i+1}\) y \(v_{i+1} = v_{i-1} - v_{i}q_{i+1} \) para cualquier \(i\),
 \[
\begin{aligned}
    \deg r_{i-1} = \deg r_{i} + \deg q_{i+1} \\
    \deg v_{i+1} = \deg v_i + \deg q_{i+1}.
\end{aligned}
\]
Ahora, por la hipótesis de inducción, \(\deg f = \deg r_{i-1} + \deg v_{i} = \deg r_{i} + \deg q_{i+1} + \deg v_{i+1} - \deg q_{i+1} = \deg r_i + \deg v_{i+1}\)
\end{proof}

Ahora, con el objetivo de poder definir la evaluación de nuestro polinomios, dado \(j \geq 0\), definimos la \textit{norma j-ésima} para cualquier \(\gamma \in \F\) de forma recursiva como sigue:
\[
N_0(\gamma) = 1
\]
\[
N_{j+1}(\gamma) = \gamma \sigma(N_{j}(\gamma)) = \gamma \sigma(\gamma)\dots\sigma^{j}(\gamma)
.\]
La noción de norma j-ésima admite también una versión para índices negativos dada por
\[
N_{-j-1}(\gamma) = \gamma \sigma^{-1}(N_{-j}(\gamma)) =  \gamma \sigma^{-1}(\gamma) \cdots \sigma^{-j}(\gamma)
.\]

En la siguiente proposición se muestran dos propiedades de esta norma que nos serán de utilidad más adelante.

\begin{proposition}
\label{prop:norm_properties}
Sean \(\gamma, \alpha, \beta \in \F\) tales que \(\beta = \alpha^{-1}\sigma(\alpha)\), entonces
\[
\begin{aligned}
&N_i(\sigma^{k}(\gamma)) = \sigma^{k}(N_i(\gamma)),\\
&N_i(\sigma^k(\beta)) = \sigma^{k}(\alpha)^{-1} \sigma^{k+i}(\alpha).
\end{aligned}\]
\end{proposition}

\begin{proof}
La primera igualdad se prueba de forma inmediata usando la propia definición de la norma, pues
\[
\begin{aligned}
N_i(\sigma^{k}(\gamma)) &= \sigma^{k}(\gamma) \sigma(\sigma^{k}(\gamma))\cdots \sigma^{i-1}(\sigma^{k}(\gamma)) \\
&= \sigma^{k}(\gamma) \sigma^{k}(\sigma(\gamma))\cdots \sigma^{k}(\sigma^{i-1}(\gamma)) = \sigma^{k}(N_i(\gamma)).
\end{aligned}
\]

Para la segunda igualdad, utilizando la igualdad anterior y de nuevo la definición de norma tenemos que
\[
\begin{aligned}
N_i(\sigma^{k}(\beta)) &= \sigma^{k}(N_i(\beta)) = \sigma^{k}\left(\alpha^{-1}\sigma(\alpha)\sigma(\alpha^{-1})\sigma^{2}(\alpha)\sigma^{2}(\alpha^{-1}) \ldots \sigma^{i}(\alpha) \right) \\
   &= \sigma^{k}(\alpha)^{-1} \sigma^{k+i}(\alpha).
\end{aligned}
.\]
\end{proof}

La \textit{evaluación la izquierda} de un polinomio no conmutativo \(f \in R\) en un \(a \in \F\) se define como el resto de la división a la derecha de \(f\) por  \(x - a\), y de forma análoga para la  \textit{evaluación a la derecha}. Estas evaluaciones nos permiten hablar de raíces a izquierda y derecha de estos polinomios. Las propiedades en general de estas son estudiadas en \cite{Lam_1988}, donde se encuentra en esencia la prueba que presentamos del siguiente lema.

\begin{lemma}
\label{lem:eval}
    Sea \(\gamma \in \F\) y  \(f = \sum_{i=0}^n f_ix^i \in R\). Entonces:
    \begin{enumerate}
    \item El resto de la división a la izquerda de \(f\) por  \(x - \gamma\) es  \(\sum_{i=0}^{n} f_i N_i(\gamma)\).
    \item El resto de la división a la derecha de \(f\) por  \(x - \gamma\) es  \(\sum_{i=0}^{n}\sigma^{-i}(f_i)N_{-i}(\gamma)\).
    \item \(N_j(\sigma^k(\gamma)) = \sigma^k(N_j(\gamma))\) para todo \(i,k\).
    \end{enumerate}
\end{lemma}
\begin{proof}
    Para demostrar \textit{i)} observamos primero un caso especial de este resultado:
\begin{equation}
\label{norm_proof}
    x^j - N_j(\gamma) \in R(x-\gamma)\ \forall j \geq 0.
\end{equation}
    Es evidente que el resultado es cierto para \(j = 0\), asi que procedemos por inducción sobre \(j\). Supongamos el resultado cierto para \(j\), entonces
\[
\begin{aligned}
x^{j+1} - N_{j+1}(\gamma) &= x^{j+1} - \sigma(N_j(\gamma))\gamma \\
&= x^{j+1} +  \sigma(N_j(\gamma))(x-\gamma) - \sigma(N_j(\gamma))x \\
&= x^{j+1} +  \sigma(N_j(\gamma))(x-\gamma) - xN_j(\gamma) \\
&= \sigma(N_j(\gamma))(x-\gamma) + x(x^{j} - N_j(\gamma)) \in R(x-\gamma)
\end{aligned}
.\]
Utilizando (\ref{norm_proof}) tenemos entonces que
\[
f - \sum_{i=0}^n f_i N_i(\gamma) = \sum_{i=0}^n f_i(x^i - N_i(\gamma)) \in R(x - \gamma)
\]
y, por la unicidad del resto de la división euclídea tenemos que \(r = \sum_{i=0}^n f_i N_i(\gamma)\).

Para la siguiente afirmación procedemos de forma similar. Veamos en primer lugar que
\begin{equation}
\label{right_norm_proof}
    x^j - N_{-j}(\gamma) \in (x-\gamma)R\ \forall j \geq 0.
\end{equation}

De nuevo, es obvio para \(j = 0\), por tanto procedemos por inducción supuesto cierto para \(j\).
\[
\begin{aligned}
x^{j+1} - N_{-j-1}(\gamma) &= x^{j+1} - \gamma\sigma^{-1}(N_j(\gamma))  \\
&= x^{j+1} +  (x-\gamma)\sigma^{-1}(N_{-j}(\gamma)) - x\sigma^{-1}(N_{-j}(\gamma)) \\
&= x^{j+1} + (x-\gamma)\sigma^{-1}(N_{-j}(\gamma)) - N_{-j}(\gamma)x \\
&= (x-\gamma)\sigma^{-1}(N_{-j}(\gamma)) + (x^{j} - N_{-j}(\gamma))x \in (x-\gamma)R
\end{aligned}
.\]
Así, usando (\ref{right_norm_proof}) vemos que
\[
f - \sum_{i=0}^n \sigma^{-i}(f_i) N_{-i}(\gamma) = \sum_{i=0}^n x^i\sigma^{-i}(f_i) - N_{-i}(\gamma)\sigma^{-i}(f_i)
\]
\[
= \sum_{i=0}^n (x^i - N_{-i}(\gamma))\sigma^{-i}(f_i) \in (x - \gamma)R
.\]
Así que por la unicidad del resto queda probado \textit{ii)}.

Para probar \textit{iii)},
    \[
    N_j(\sigma^k(\gamma)) = \sigma^k(\gamma)\sigma^{k+1}(\gamma)\cdots\sigma^{k+j-1}(\gamma)
\]
\[
    = \sigma^k(\gamma \sigma(\gamma)\dots\sigma^{j-1}(\gamma)) = \sigma^k(N_j(\gamma))
    .\]
\end{proof}

\chapter{Algoritmo para códigos cíclico sesgados}%
\label{chap:conceptos_básicos_sobre_códigos_lineales}

\begin{lemma}
    Sea \(L\) un cuerpo, \(\sigma\) un automorfismo de \(L\) de orden finito \(n\), y \(K = L^\sigma\) el subcuerpo invariante bajo \(\sigma\). Sea  \(\{a_0, \dots, a_{n-1}\) una \(K\)-base de \(L\). Entonces, para todo \(t \leq n\), y cada subconjunto \(k_0 < k_1 < \dots < k_{t-1} \subset \{0, 1, \dots, n-1\}\)
    \[
    \begin{vmatrix}
        \alpha_{k_0} & \alpha_{k_1} & \dots & \alpha_{k_{t -1}} \\
        \sigma^{t-1}(\alpha_{k_0}) & \sigma^{t-1}(\alpha_{k_1}) & \dots & \sigma^{t-1}(\alpha_{k_{t-1}}) \\
        \vdots & \vdots & \ddots & \vdots \\
        \sigma^{t-1}(\alpha_{k_0}) & \sigma^{t-1}(\alpha_{k_1}) & \dots & \sigma^{t-1}(\alpha_{k_{t-1}})
    \end{vmatrix}
    \neq 0
    .\]
\end{lemma}

\begin{proofs}
    Realizaremos la prueba por inducción sobre \(t\). El caso \(t = 1\) se cumple trivialmente. Por tanto, supongamos que el lema se cummple para un cierto  \(t \geq 1\). Tenemos que comprobar que, para toda matriz \((t+1) \times (t+1)\)
    \[
    \Delta =
    \begin{pmatrix}
        \alpha_{k_0} & \alpha_{k_1} & \dots & \alpha_{k_{t -1}} \\
        \sigma^{t-1}(\alpha_{k_0}) & \sigma^{t-1}(\alpha_{k_1}) & \dots & \sigma^{t-1}(\alpha_{k_{t-1}}) \\
        \vdots & \vdots & \ddots & \vdots \\
        \sigma^{t}(\alpha_{k_0}) & \sigma^{t}(\alpha_{k_1}) & \dots & \sigma^{t}(\alpha_{k_{t}})

    \end{pmatrix}
    .\]
el determinante \(|\Delta|\) es distinto de cero. Supongamos por el contrario que \(|\Delta| = 0\). Por la hipótesis de inducción tenemos que las primeras \(t\) columnas de \(\Delta\) son linealmente independientes, luego existen \(a_0, \dots, a_{t-1} \in L\) tales que

\[
(a_{k_t}, \sigma^{t-1}(a_{k_t}, \dots, \sigma^t(a_{k_t}))) = \sum_{j=0}^{t-1} a_j(\alpha_{k_j}, \sigma^{t-1}(\alpha_{k_j}), \dots, \sigma^t(\alpha_{k_j}))
.\]

Es decir, \(a_0, \dots, a_{t-1}\) satisfacen el sistema lineal

\begin{equation}
\label{linear_system}
\begin{cases}
    \alpha_{k_t} = a_0\alpha_{k_0} + \cdots + a_{t-1}\alpha_{k_{t-1}} \\
    \sigma^{t-1}(\alpha_{k_t}) = a_0\sigma^{t-1}(\alpha_{k_0}) + \cdots + a_{t-1}\sigma^{t-1}(\alpha_{k_{t-1}}) \\
    \vdots \\
    \sigma^t(\alpha_{k_t}) = a_0\sigma^t(\alpha_{k_0}) + \cdots + a_{t-1}\sigma^t(\alpha_{k_{t-1}})
\end{cases}
.
\end{equation}

Para cada \(j = 0, \dots, t-1\), restamos en (\ref{linear_system}) la ecuación \(j+1\) transformada por \(\sigma^{-1}\) a la ecuación \(j\). Esto produce el siguiente sistema lineal homogéneo

\begin{equation}
\label{hom_liner_system}
\begin{cases}
0 = (a_0 - \sigma^{-1}(a_0))\alpha_{k_0} + \cdots + (a_{t-1} - \sigma^{-1}(a_{t-1}))\alpha_{k_{t-1}} \\
0 = (a_0 - \sigma^{-1}(a_0))\sigma^{t-1}(\alpha_{k_0}) + \cdots + (a_{t-1} - \sigma^{-1}(a_{t-1}))\sigma^{t-1}(\alpha_{k_{t-1}}) \\
\vdots \\
0 = (a_0 - \sigma^{-1}(a_0))\sigma^{t-1}(\alpha_{k_0}) + \cdots + (a_{t-1} - \sigma^{-1}(a_{t-1}))\sigma^{t-1}(\alpha_{k_{t-1}}) \\
\end{cases}
.
\end{equation}

La matriz de coeficientes de (\ref{hom_liner_system}) es no singular, por la hipótesis de inducción, así que tenemos que para todo \(j = 0, \dots, t-1\), \(a_j - \sigma^{-1}(a_j) = 0\), y por tanto \(a_0, \dots, a_{t-1} \in K\). Como consecuencia, la ecuación (\ref{linear_system}) establece una dependencia lineal sobre  \(K\) de la  \(K\)-base  \(\{\alpha_0, \dots, \alpha{n-1}\), creando una contradicción. Por tanto, \(|\Delta| \neq 0\) y el resultado queda demostrado.
\end{proofs}

\begin{lemma}
    Sea \(\alpha \in \F\) tal que \(\{\alpha, \sigma(\alpha), \dots, \sigma^{n-1}(\alpha)\) sea una base de \(\F\) como  \(\F^\sigma\)-espacio vectorial. Fijemos  \(\beta = \alpha^{-1}\sigma(\alpha)\). Para todo subconjunto \(T = \{t_1 < t_2 < \cdots < t_m\} \subset \{0, 1, \dots, n-1\}\), los polinomios
    \[
    g^l = [x - \sigma^{t_1}(\beta), x - \sigma^{t_2}(\beta), \dots, x - \sigma^{t_m}(\beta)]_l
    \]
y
    \[
    g^r = [x - \sigma^{t_1}(\beta^{-1}), x - \sigma^{t_2}(\beta^{-1}), \dots, x - \sigma^{t_m}(\beta^{-1})]_r
    \]
tienen grado \(m\). Por tanto, si \(x - \sigma^s(\beta) |_r g^l\) o \(x - \sigma^s(\beta^{-1}) |_l g^r\), entonces \(s \in T\).
\end{lemma}

\begin{proofs}
    Supongamos que \(\deg g^l < m\), así que \(g^l = \sum_{i=0}^{m-1} g_i x^i\). Como \(g\) es un múltiplo a la izquierda de \(x - \sigma^{t_j}(\beta)\) para todo \(1 \leq j \leq m\), por el lema (TODO) tenemos que
    \[
    \sum_{i=0}^{m-1}g_i N_i(\sigma^{t_j}(\beta)) = 0 \text{ para todo } 1 \leq j \leq m
    .\]

Esto es un sistema lineal homogéneo cuya matriz de coeficientes es la transpuesta de
\end{proofs}


% Añade sección de referencias al final del documento.
% Selecciona un estilo de cita.
\bibliographystyle{alpha}
% En research.bib están las entradas de los artículos que citamos.
% Podemos cambiar el nombre del archivo aquí.
\bibliography{research}

\end{document}
